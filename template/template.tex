%%
%% This is file `sample-manuscript.tex',
%% generated with the docstrip utility.
%%
%% The original source files were:
%%
%% samples.dtx  (with options: `manuscript')
%%
%% IMPORTANT NOTICE:
%%
%% For the copyright see the source file.
%%
%% Any modified versions of this file must be renamed
%% with new filenames distinct from sample-manuscript.tex.
%%
%% For distribution of the original source see the terms
%% for copying and modification in the file samples.dtx.
%%
%% This generated file may be distributed as long as the
%% original source files, as listed above, are part of the
%% same distribution. (The sources need not necessarily be
%% in the same archive or directory.)
%%
%% Commands for TeXCount
%TC:macro \cite [option:text,text]
%TC:macro \citep [option:text,text]
%TC:macro \citet [option:text,text]
%TC:envir table 0 1
%TC:envir table* 0 1
%TC:envir tabular [ignore] word
%TC:envir displaymath 0 word
%TC:envir math 0 word
%TC:envir comment 0 0
%%
%%
%% The first command in your LaTeX source must be the \documentclass command. This is the generic manuscript mode required for submission and peer review.
%\documentclass[manuscript,screen,review]{acmart}
\documentclass[acmsmall]{acmart}
%% To ensure 100% compatibility, please check the white list of
%% approved LaTeX packages to be used with the Master Article Template at
%% https://www.acm.org/publications/taps/whitelist-of-latex-packages
%% before creating your document. The white list page provides
%% information on how to submit additional LaTeX packages for
%% review and adoption.
%% Fonts used in the template cannot be substituted; margin
%% adjustments are not allowed.

%%
%% \BibTeX command to typeset BibTeX logo in the docs
\AtBeginDocument{%
  \providecommand\BibTeX{{%
    \normalfont B\kern-0.5em{\scshape i\kern-0.25em b}\kern-0.8em\TeX}}}

%% Rights management information.  This information is sent to you
%% when you complete the rights form.  These commands have SAMPLE
%% values in them; it is your responsibility as an author to replace
%% the commands and values with those provided to you when you
%% complete the rights form.
%\setcopyright{acmlicensed}
\copyrightyear{2023}
%\acmYear{2018}
%\acmDOI{XXXXXXX.XXXXXXX}

\usepackage{makecell}
\usepackage{listings}
\usepackage{algorithm}
\usepackage{algpseudocode}

%% \todo{} command.
%
% Outputs red TODOs in the document. Requires \usepackage{color}.
%
% Usage: \todo{Document the TODO command.}
%
% Comment out second line to disable.
\newcommand{\todo}[1]{}
\renewcommand{\todo}[1]{{\color{red} TODO: {#1}}}

% Bitwise operations for formulas
\newcommand*\BitAnd{\mathbin{\&}}
\newcommand*\BitOr{\mathbin{|}}
%\newcommand*\BitXor{\mathbin{^}}
%\newcommand*\BitXor{\mathbin{\wedge}}
\newcommand*\BitXor{\mathbin{\hat{}}}
% \ll and \gg look a little "tight" to me visually
%\newcommand*\ShiftLeft{\ll}
%\newcommand*\ShiftRight{\gg}
\newcommand*\ShiftLeft{<<}
\newcommand*\ShiftRight{>>}
\newcommand*\BitNeg{\ensuremath{\mathord{\sim}}}

% Macros that will let me reconfigure notation later for the priority and action id
% of a rule.
\newcommand{\prio}[1]{}
\renewcommand{\prio}[1]{\ensuremath{\mathtt{prio}({#1})}}
\newcommand{\actionid}[1]{}
\renewcommand{\actionid}[1]{\ensuremath{\mathtt{actionid}({#1})}}
\newcommand{\fullmask}[1]{}
\renewcommand{\fullmask}[1]{\ensuremath{\mathtt{mask}({#1})}}
\newcommand{\prefixmask}[2]{}
\renewcommand{\prefixmask}[2]{\ensuremath{\mathtt{pmask}({#1},{#2})}}

%% These commands are for a PROCEEDINGS abstract or paper.
% \acmConference[Conference acronym 'XX]{Make sure to enter the correct
%   conference title from your rights confirmation emai}{June 03--05,
%   2018}{Woodstock, NY}
%
%  Uncomment \acmBooktitle if th title of the proceedings is different
%  from ``Proceedings of ...''!
%
%\acmBooktitle{Woodstock '18: ACM Symposium on Neural Gaze Detection,
%  June 03--05, 2018, Woodstock, NY}

%% These commands are for a JOURNAL article.
%\acmJournal{JACM}
%\acmVolume{37}
%\acmNumber{4}
%\acmArticle{111}
%\acmMonth{8}

%\acmISBN{978-1-4503-XXXX-X/18/06}


%%
%% Submission ID.
%% Use this when submitting an article to a sponsored event. You'll
%% receive a unique submission ID from the organizers
%% of the event, and this ID should be used as the parameter to this command.
%%\acmSubmissionID{123-A56-BU3}

%%
%% For managing citations, it is recommended to use bibliography
%% files in BibTeX format.
%%
%% You can then either use BibTeX with the ACM-Reference-Format style,
%% or BibLaTeX with the acmnumeric or acmauthoryear sytles, that include
%% support for advanced citation of software artefact from the
%% biblatex-software package, also separately available on CTAN.
%%
%% Look at the sample-*-biblatex.tex files for templates showcasing
%% the biblatex styles.
%%

%%
%% The majority of ACM publications use numbered citations and
%% references.  The command \citestyle{authoryear} switches to the
%% "author year" style.
%%
%% If you are preparing content for an event
%% sponsored by ACM SIGGRAPH, you must use the "author year" style of
%% citations and references.
%% Uncommenting
%% the next command will enable that style.
%%\citestyle{acmauthoryear}

%%
%% end of the preamble, start of the body of the document source.
\begin{document}

%%
%% The "title" command has an optional parameter,
%% allowing the author to define a "short title" to be used in page headers.
\title{todo title here}

%%
%% The "author" command and its associated commands are used to define
%% the authors and their affiliations.
%% Of note is the shared affiliation of the first two authors, and the
%% "authornote" and "authornotemark" commands
%% used to denote shared contribution to the research.
\author{todo author}
%\authornote{Both authors contributed equally to this research.}
\email{todo email}
\orcid{1234-5678-9012}


%%
%% By default, the full list of authors will be used in the page
%% headers. Often, this list is too long, and will overlap
%% other information printed in the page headers. This command allows
%% the author to define a more concise list
%% of authors' names for this purpose.
\renewcommand{\shortauthors}{todo Trovato and Tobin, et al.}

%%
%% The abstract is a short summary of the work to be presented in the
%% article.
\begin{abstract}
\todo{todo Abstract here.}
\end{abstract}

%%
%% The code below is generated by the tool at http://dl.acm.org/ccs.cfm.
%% Please copy and paste the code instead of the example below.
%%
%\begin{CCSXML}
%<ccs2012>
% <concept>
%  <concept_id>00000000.0000000.0000000</concept_id>
%  <concept_desc>Do Not Use This Code, Generate the Correct Terms for Your Paper</concept_desc>
%  <concept_significance>500</concept_significance>
% </concept>
% <concept>
%  <concept_id>00000000.00000000.00000000</concept_id>
%  <concept_desc>Do Not Use This Code, Generate the Correct Terms for Your Paper</concept_desc>
%  <concept_significance>300</concept_significance>
% </concept>
% <concept>
%  <concept_id>00000000.00000000.00000000</concept_id>
%  <concept_desc>Do Not Use This Code, Generate the Correct Terms for Your Paper</concept_desc>
%  <concept_significance>100</concept_significance>
% </concept>
% <concept>
%  <concept_id>00000000.00000000.00000000</concept_id>
%  <concept_desc>Do Not Use This Code, Generate the Correct Terms for Your Paper</concept_desc>
%  <concept_significance>100</concept_significance>
% </concept>
%</ccs2012>
%\end{CCSXML}

%\ccsdesc[500]{Do Not Use This Code~Generate the Correct Terms for Your Paper}
%\ccsdesc[300]{Do Not Use This Code~Generate the Correct Terms for Your Paper}
%\ccsdesc{Do Not Use This Code~Generate the Correct Terms for Your Paper}
%\ccsdesc[100]{Do Not Use This Code~Generate the Correct Terms for Your Paper}

%%
%% Keywords. The author(s) should pick words that accurately describe
%% the work being presented. Separate the keywords with commas.
%\keywords{Do, Not, Us, This, Code, Put, the, Correct, Terms, for,
%  Your, Paper}

%\received{20 February 2007}
%\received[revised]{12 March 2009}
%\received[accepted]{5 June 2009}

%%
%% This command processes the author and affiliation and title
%% information and builds the first part of the formatted document.
\maketitle

\section{Introduction}

Try citing a paper with a plus sign in its BibTeX identifier~\cite{PS1985}.


\section{Comparison of algorithms for normal packet classification problem}

Notation:
\begin{itemize}
  \item $n$ is the number of rules.
  \item $d$ is the number of fields, or dimensions.
  \item $w$ is the number of bits in the largest field.
  \item $W$ is the word size in the RAM model of computation.
  \item $t$ is the number of matching rules.  Algorithms that includes this in the search time return all matching rules.
  \item $H(n,w)$ is the time to perform a hash lookup in a table with up to $n$ elements, each $w$ bits wide.  Often this is considered to be $O(1)$ in papers that use hashing.
  \item $L(n,w)$ is the time for any longest prefix match algorithm on $n$ prefixes of a field that is $w$ bits wide.
  \item $k$ is a parameter chosen by the implementer
\end{itemize}

Table~\ref{tab:summary1} summarizes some memory and run-time properties of several algorithms for the normal packet classification problem with arbitrary dimension $d$.

\begin{table*}
  \caption{Algorithms for normal packet classification problem}
  \label{tab:summary1}
  \begin{tabular}{lccccl}
    \toprule

      Algorithm
    & \makecell{Match \\ kinds}
    & Search time
    & Memory
    & Construction time
    & Source \\

    \midrule

      0
    & \makecell{ternary, \\ range}
    & $O(n)$
    & $O(n)$
    & $O(n)$
    & linear search
    \\

      bit vector
    & range
    & $O(d(n/k) + d \log(n))$
    & $O(n^2)$
    & $O(n^2)$ ?
    & \cite[Sec. 4]{LS1998}
    \\

      1
    & range
    & $O((\log n)^{2d-1} + t)$
    & $O(n (\log n)^{d-1} )$
    & $O(n (\log n)^d )$
    & \cite[Thm. 3.2]{Edel1983a}
    \\

      2
    & range
    & $O((\log n)^{2d-1} + t)$
    & $O(n (\log n)^{2d-1} )$
    & $O(n (\log n)^{2d-1} )$
    & \cite[Cor. 2.2]{Edel1983a}
    \\

      3
    & range
    & $O(d \log n)$
    & $O(n^d)$
    & $O(n^d)$
    & \cite[Sec. 2.3]{PS1985}
    \\

      4
    & prefix
    & $O(d L(n,w))$
    & $O(n^d)$
    & $O(n^d)$
    & \cite[Sec. 2.3]{PS1985}
    \\

      \makecell{CNRT\footnotemark{} \\ ($d \geq 3$)}
    & range
    & $O(\log^{2}_{W} n \cdot \log^{d-3} n + t)$
    & $O(n \log^{d-3} n)$
    & ?
    & \cite[Thm. 6]{CNRT2022}
    \\

  \bottomrule
\end{tabular}
\end{table*}

\footnotetext{Chan et al call the problem of interest to us the rectangle stabbing problem}


\subsection{Definitions of classification problem, match kinds, match criteria}

%%
%% The acknowledgments section is defined using the "acks" environment
%% (and NOT an unnumbered section). This ensures the proper
%% identification of the section in the article metadata, and the
%% consistent spelling of the heading.
\begin{acks}
todo acknowledgements
\end{acks}

%%
%% The next two lines define the bibliography style to be used, and
%% the bibliography file.
\bibliographystyle{ACM-Reference-Format}
\bibliography{template}

%%
%% If your work has an appendix, this is the place to put it.
\appendix

\section{Proofs}

\end{document}
\endinput
%%
%% End of file `sample-manuscript.tex'.
