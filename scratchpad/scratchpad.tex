\documentclass[a4paper]{article}

\usepackage[pages=all, color=black, position={current page.south}, placement=bottom, scale=1, opacity=1, vshift=5mm]{background}
%\SetBgContents{
%	\tt This work is shared under a \href{https://creativecommons.org/licenses/by-sa/4.0/}{CC BY-SA 4.0 license} unless otherwise noted
%}      % copyright

\usepackage[margin=1in]{geometry} % full-width

% AMS Packages
\usepackage{amsmath}
\usepackage{amsthm}
\usepackage{amssymb}

% Unicode
\usepackage[utf8]{inputenc}
\usepackage{hyperref}
\hypersetup{
	unicode,
%	colorlinks,
%	breaklinks,
%	urlcolor=cyan, 
%	linkcolor=blue, 
	pdfauthor={Author One, Author Two, Author Three},
	pdftitle={A simple article template},
	pdfsubject={A simple article template},
	pdfkeywords={article, template, simple},
	pdfproducer={LaTeX},
	pdfcreator={pdflatex}
}

% Natbib
\usepackage[sort&compress,numbers,square]{natbib}
\bibliographystyle{mplainnat}

% Theorem, Lemma, etc
\theoremstyle{plain}
\newtheorem{theorem}{Theorem}
\newtheorem{corollary}[theorem]{Corollary}
\newtheorem{lemma}[theorem]{Lemma}
\newtheorem{claim}{Claim}[theorem]
\newtheorem{axiom}[theorem]{Axiom}
\newtheorem{conjecture}[theorem]{Conjecture}
\newtheorem{fact}[theorem]{Fact}
\newtheorem{hypothesis}[theorem]{Hypothesis}
\newtheorem{assumption}[theorem]{Assumption}
\newtheorem{proposition}[theorem]{Proposition}
\newtheorem{criterion}[theorem]{Criterion}
\theoremstyle{definition}
\newtheorem{definition}[theorem]{Definition}
\newtheorem{example}[theorem]{Example}
\newtheorem{remark}[theorem]{Remark}
\newtheorem{problem}[theorem]{Problem}
\newtheorem{principle}[theorem]{Principle}

\usepackage{graphicx, color}
\graphicspath{{fig/}}

%\usepackage[linesnumbered,ruled,vlined,commentsnumbered]{algorithm2e} % use algorithm2e for typesetting algorithms
\usepackage{algorithm, algpseudocode} % use algorithm and algorithmicx for typesetting algorithms
\usepackage{mathrsfs} % for \mathscr command

% Author info
\title{Andy's science scratch pad}
\author{J. Andrew Fingerhut (\texttt{andy.fingerhut@gmail.com})}

\date{
        May 31, 2025
%	\today
}

\newcommand{\ihat}{\textbf{i}}
\newcommand{\jhat}{\textbf{j}}
\newcommand{\khat}{\textbf{k}}
\newcommand{\rhat}{\hat{\textbf{r}}}
\newcommand{\vect}[1]{\textbf{#1}}

\begin{document}
\maketitle

%\begin{abstract}
%  todo: abstract here
%\end{abstract}

\tableofcontents

\section{Introduction}
\label{sec:intro}

This document is a place to write up little bits on science.

Some notation:

$\ihat$ is the unit vector from left to right.
$\jhat$ is the unit vector upwards.
$\khat$ is the unit vector pointed out of the page toward the reader.

$\gamma = 1/\sqrt{1-v^2/c^2}$ is the Lorentz factor.


\section{Electromagnetic force between two point charges at rest relative to each other}
\label{sec:twocharges}

Scenario 1: There are two point charges $a$ and $b$ both with charge
$q$ at rest relative to each other at a distance $r$ apart (see
Figure~\ref{fig:two-charges-at-rest}).  They are at rest relative
to us.  In this case they both experience a force directly away from
the other due to electric repulsion.  There is no magnetic force, as
both charges are at rest so there are no magnetic fields.

\begin{figure}[ht]
	\centering
	\includegraphics[width=0.6\textwidth]{two-charges-at-rest-cropped.pdf}
	\caption{Two point charges at rest}
	\label{fig:two-charges-at-rest}
\end{figure}


Scenario 2: The same as scenario 1, but both charges are moving with
constant velocity $v$ in the upwards direction (see
Figure~\ref{fig:two-charges-moving}).  Since they are moving they
create magnetic fields.

\begin{figure}[ht]
	\centering
	\includegraphics[width=0.4\textwidth]{two-charges-moving-cropped.pdf}
	\caption{Two point charges moving at same constant velocity}
	\label{fig:two-charges-moving}
\end{figure}

Questions:
\begin{itemize}
  \item What is the net force on charge $b$ in each scenario?
  \item Is it the same in both scenarios, or different?
  \item Why?
\end{itemize}


\subsection{Scenario 1: Both charges at rest}

As mentioned before, there is no current or motion of any charges in
this scenario, so no magnetic fields.  The electric repulsion force on
charge $b$ is easily calculated from Coulomb's Law~\cite{CoulombsLaw}.
Charge $b$ is to the right of charge $a$, so the direction of the force is
$\ihat$, away from charge $a$.

\begin{equation}
\vect{E}_1 = \frac{1}{4 \pi \epsilon_0} \frac{q}{r^2} \ihat \label{eq:E1}
\end{equation}

\begin{equation}
\vect{B}_1 = 0
\end{equation}

\begin{equation}
\vect{F}_1 = q(\vect{E}_1 + \vect{v} \times \vect{B}_1)
           = q \vect{E}_1   \label{eq:F1}
\end{equation}


\subsection{Scenario 2: Both charges with equal and constant velocity upwards}

\subsubsection{Scenario 2 calculated by electromagnetic field equations from Griffiths}

The Wikipedia page on the Biot-Savart
Law~\cite{EMFieldFromPointCharge} has a subsection titled ``Point
charge at constant velocity'' that says:

\begin{quote}
the Biot–Savart law applies only to steady currents and a point charge
moving in space does not constitute a steady current
\end{quote}

I will thus use the equations in that section to calculate the
electric and magnetic fields here.  The relevant parts of the
Wikipedia page are copied below.

\begin{quote}
In the case of a point charged particle $q$ moving at a constant
veclocity $\vect{v}$, Maxwell's equations give the following
expression for the electric field and magnetic field:
\end{quote}
\begin{align}
\vect{E} & = \frac{q}{4 \pi \epsilon_0} \frac{1-\beta^2}{(1-\beta^2 \sin^2 \theta)^{3/2}} \frac{{\rhat}'}{|r'|^2} \label{eq:EforPtChg} \\
\vect{B} & = \frac{1}{c^2} \vect{v} \times \vect{E} \label{eq:BforPtChg}
\end{align}
where:
\begin{itemize}
    \item ${\rhat}'$ is the unit vector pointing from the current
      (non-retarded) position of the particle to the point at which
      the field is being measured,
    \item $\beta = v/c$ is the speed in units of $c$, and
    \item $\theta$ is the angle between $\vect{v}$ and ${\rhat}'$.
\end{itemize}

The equations above appear to be identical to equations (10.75) and
(10.76) in Griffiths~\cite{Griffiths1998}.  Griffiths comments on the formula for the electric field:

\begin{quote}
Notice that $\vect{E}$ points along the line from the {\em present}
position of the particle.  This is an extraordinary coincidence, since
the ``message'' came from the retarded position.  Because of the
$\sin^2 \theta$ in the denominator, the field of a fast-moving charge
is flattened out like a pancake in the direction perpendicular to the
motion (Fig. 10.10).  In the forward and backward directions
$\vect{E}$ is reduced by a factor $(1 - v^2/c^2)$ relative to the
field of a charge at rest; in the perpendicular direction it is
{\em enhanced} by a factor $1 / \sqrt{ 1 - v^2/c^2}$.
\end{quote}

Calculation: To get the force on charge $b$, we first calculate the
$\vect{E}$ and $\vect{B}$ fields at the position of charge $b$.

Charge $b$ is directly to the right of charge $a$, so ${\rhat}' = \ihat$
and $\theta = 90^{\circ}$.

\begin{align}
\vect{E}_2
  & = \frac{q}{4 \pi \epsilon_0} \frac{1-\beta^2}{(1-\beta^2 \sin^2 \theta)^{3/2}} \frac{{\rhat}'}{|r'|^2} & & \text{${\rhat}' = \ihat$, $|r'| = r$, $\theta=90^{\circ}$, simplify fraction} \nonumber \\
  & = \frac{q}{4 \pi \epsilon_0} \frac{1}{(1-\beta^2)^{1/2}} \frac{\ihat}{r^2} & & \text{part of this is $\gamma$, by~\eqref{eq:E1} the rest is $\vect{E}_1$} \nonumber \\
  & = \gamma \vect{E}_1 \label{eq:E2value}
\end{align}

Note that $\vect{E}_2$ being $\gamma$ times larger than $\vect{E}_1$
is consistent with the comment from Griffiths above: ``in the
perpendicular direction it ($\vect{E}$) is {\em enhanced} by a factor
$1 / \sqrt{ 1 - v^2/c^2}$''.

\begin{align*}
\vect{F}_2
  & = q (\vect{E}_2 + \vect{v} \times \vect{B}_2)   & & \text{replace $\vect{B}_2$ with \eqref{eq:BforPtChg}} \\
  & = q (\vect{E}_2 + \vect{v} \times (\frac{1}{c^2} \vect{v} \times \vect{E}_2))  & & \vect{v} \times \vect{E}_2 = - v E_2 \khat \\
  & = q (\vect{E}_2 - \frac{v E_2}{c^2} \vect{v} \times \khat)  & & \vect{v} \times \khat = v \ihat \\
  & = q (\vect{E}_2 - \frac{v^2 E_2}{c^2} \ihat) \\
  & = q (1 - \frac{v^2}{c^2}) \vect{E}_2 \\
  & = \frac{q \vect{E}_2}{\gamma^2} & & \text{by~\eqref{eq:E2value} \ } \vect{E}_2 = \gamma \vect{E}_1 \\
  & = \frac{q \vect{E}_1}{\gamma} & & \text{by~\eqref{eq:F1} \ } \vect{F}_1 = q \vect{E}_1 \\
  & = \frac{\vect{F}_1}{\gamma}
\end{align*}

Thus $\vect{F}_2$ differs from $\vect{F}_1$ by a factor of $\gamma$.

TODO: Why?

I do not know how to check the answer below, but it appears that three
of the answers to an on-line question similar to
mine~\cite{PhysicsSEIsLorentzForceFrameIndependent} say that the
Lorentz force formula $\vect{F} = q(\vect{E} + \vect{v} \times
\vect{B})$ is {\em not} invariant in all inertial frames, but perhaps
a slightly modified version of that formula is invariant between
different inertial frames.  I quote one such answer below:

\begin{quote}
Just for completeness if permitted: Following Section 3.1 from the
book ``Gravitation'' of Misner, Thorne, and Wheeler the truly (at all
speeds) frame independent force is $\frac{dP}{d \tau} = \gamma (E + v
\times B)$ (in fact this is only the spacial component of the four
force).  $\tau$ is proper time and $\gamma$ the well-known Lorentz
Factor. -- Kurt G. Aug 28, 2021
\end{quote}


\subsubsection{Scenario 2 calculated by Heaviside-Feynman formula}

The Wikipedia page on Jefimenko's Equations~\cite{JefimenkosEquations}
has a subsection titled ``Heaviside-Feynman formula'' that gives
equations for the electric and magnetic field at a point due to a
single moving point charge.

\begin{align}
\vect{E} & = \frac{-q}{4 \pi \epsilon_0}
             \left[
               \frac{\vect{e}_{r'}}{r'^2}
               + \frac{r'}{c} \frac{d}{dt} \left( \frac{\vect{e}_{r'}}{r'^2} \right)
               + \frac{1}{c^2} \frac{d^2}{dt^2} \vect{e}_{r'}
             \right]
             \label{eq:HF-EforPtChg} \\
\vect{B} & = - \vect{e}_{r'} \times \frac{\vect{E}}{c}
             \label{eq:HF-BforPtChg}
\end{align}

Here $\vect{e}_{r'}$ is a unit vector pointing from the observer to
the charge and $r'$ is the distance between observer and charge.
Since the electromagnetic field propagates at the speed of light, both
of these quantities are evaluated at the retarted time $t - r'/c$.

I believe ``observer'' above means ``the position for which we are
calculating $E$ and $B$ fields''.

Assume here that the point charges are kept at distance $r$ apart from
each other, always horizontally, e.g. because they are connected by a
stiff insulating rod.  This simplifies our job of calculating $E$,
because then $\vect{e}_{r'}$ and $r'$ are unchanging over time, and
their derivates are thus 0.

We want to calculate $r'$ as the vector from the position of charge
$b$ to the position where charge $a$ was when it emitted an electric
field propagated at speed $c$ to $b$.  See
Figure~\ref{fig:retarded-position}.

\begin{figure}[ht]
	\centering
	\includegraphics[width=0.5\textwidth]{retarded-position-cropped.pdf}
	\caption{The retarded position of charge $a$ from charge $b$}
	\label{fig:retarded-position}
\end{figure}

Solve for $t$ using Pythagorean theorem since $r$ and $v$ are known
constants:
\begin{align*}
r^2 + (vt)^2 & = (ct)^2 \\
t^2 (c^2 - v^2) & = r^2 \\
t^2 & = \frac{r^2}{c^2 - v^2} \\
t & = \frac{r}{\sqrt{c^2-v^2}} \\
  & = \frac{r}{c \sqrt{1 - v^2/c^2}} \\
  & = \gamma r / c
\end{align*}

This gives us $r' = ct = \gamma r$, and $\vect{e}_{r'}$ is:

\begin{align*}
\vect{e}_{r'} & = \frac{-r \ihat - (\gamma r v / c) \jhat}{\gamma r} \\
  & = - \frac{1}{\gamma} \ihat - \frac{v}{c} \jhat
\end{align*}

Plugging in this value for $\vect{e}_{r'}$ into
Equation~\eqref{eq:HF-EforPtChg} gives:

\begin{align*}
\vect{E}_3 & = \frac{q}{4 \pi \epsilon_0}
             \left[
               \frac{\frac{1}{\gamma} \ihat + \frac{v}{c} \jhat}{\gamma^2 r^2}
             \right]
\end{align*}

Note that $\vect{E}_3$ is parallel to $\vect{e}_{r'}$, thus
$\vect{B}_3$ from Equation~\eqref{eq:HF-BforPtChg} is 0.

This gives the force on charge $b$ as:
\begin{align*}
\vect{F}_3
  & = q (\vect{E}_3 + \vect{v} \times \vect{B}_3) \\
  & = q \vect{E}_3
\end{align*}
The direction of $\vect{F}_3$ is different than $\vect{F}_1$ and $\vect{F}_2$.
Below is the relative magnitude of $\vect{E}_3$ to $\vect{E}_1$:
\begin{align*}
E_3 & = \frac{1}{\gamma^2} E_1 \\
F_3 & = \frac{1}{\gamma^2} F_1
\end{align*}

TODO: It seems {\em very} odd to me that $\vect{B}_3 = 0$.

After Feynman explains what the retarded direction and distance
$\vect{r'}$ is, he says~\cite{FeynmanLecturesVolICh28}:
\begin{quote}
That would be easy enough to understand, too, but it is also
wrong.  The whole thing is much more complicated.
\end{quote}
Unfortunately there are no footnotes or citation to explain what he
meant by this.


\section{Simple scenarios in special relativity and Lorentz Ether Theory}
\label{sec:sr}

Definitions of some terms:
\begin{align}
  \beta_B & = v_B / c \label{defn:betaB} \\
  \gamma_B & = \frac{1}{\sqrt{1 - \beta_B^2}} \label{defn:gammaB} \\
  D_B & = \sqrt{ \frac{1 + \beta_B}{1 - \beta_b} } \label{defn:DB}
\end{align}


\subsection{Special Relativity Scenario 1: Two entities moving away from each other at constant velocity}
\label{sec:scen1}

$A$ is at rest.  $B$ is moving at constant velocity $v_B$ away from $A$.


\subsubsection{$B$ sends periodic pulses to $A$}
\label{sec:scen1BtoA}

$B$ uses his local clock to time the sending of radio pulses to $A$,
sending pulses once every time interval $T$.
At what period does $A$ receive the pulses?

For simplicity of calculations, we will assume that $B$ passed $A$'s
position at time 0, and $A$ and $B$ synchronized their clocks at that
time.  However, note that $A$ will receive the pulses at the same
period in this scenario, regardless of whether they ever synchronized
their clocks.

Define $t_A(n)$ to be the time on A's clock when the $n$-th pulse is
transmitted by $B$, and $t_B(n)$ to be the time on B's clock when it
transmits the $n$-th pulse.

$t_B(n) = nT$ by the setup of the experiment.

By the assumptions of special relativity, $A$ deduces that $B$'s clock
is running $\gamma_B$ times slower than $A$'s clock.  Note: $A$ cannot
directly observe $B$'s clock, as it is too far away.  From this $A$
also assumes:
\begin{equation}
  t_A(n) = \gamma_B nT  \label{eqn:scen1timeA}
\end{equation}
$B$ is a distance $v_B t_A(n)$ away from $A$ at that time.

Also by the assumptions of special relativity, $A$ deduces that $B$'s
pulse signal will propagate at the one-way speed $c$.  The pulse will
thus take time $v_B t_A(n)/c$ to propagate to $A$.

$A$'s clock thus shows time $t_A(n) + v_B t_A(n)/c$ when the $n$-th
pulse arrives at $A$.  With a little algebra:
\begin{align*}
t_A(n) + v_B t_A(n)/c
  & = (1 + v_B/c) t_A(n) & & \text{rearrangement by algebra} \\
  & = nT (1 + \beta_{B}) \gamma_{B} & & \text{Defn.~\eqref{defn:betaB} and Eqn.~\eqref{eqn:scen1timeA}} \\
  & = nT \frac{1 + \beta_B}{\sqrt{ 1 - \beta_{B}^2 }} & & \text{Defn.~\eqref{defn:gammaB}} \\
  & = nT \sqrt{ \frac{1 + \beta_B}{1 - \beta_B} } & & \text{algebra} \\
  & = nT D_B & & \text{algebra} & & \text{Defn.~\eqref{defn:DB}}
\end{align*}
The pulses arrive at $A$ with a period of $D_B T$.


\subsubsection{$A$ sends periodic pulses to $B$}
\label{sec:scen1AtoB}

Now $A$ uses his local clock to time the sending of radio pulses to $B$,
sending pulses once every time interval $T$.
At what period does $B$ receive the pulses?

While we could switch perspectives to $B$'s inertial frame, we will
not.  Instead, we are going to do all of the calculations in $A$'s
inertial frame.

The definitions of Section~\ref{sec:sr} remain the same here, but note
that the values of $t_A(n)$ and $t_B(n)$ here are {\em different} than
those in Section~\ref{sec:scen1BtoA}.

Define $t_A(n) = nT$ to be the time on A's clock when it transmits its
$n$-th pulse.

At all times $t_A$, $B$ is a distance $v_B t_A$ away from $A$.  $A$
assumes by special relativity that its pulse propagates with one-way
speed $c$ to $B$.  The pulse's position at time $t_A \geq nT$ is
$c(t_A - nT)$.

On $A$'s clock, $A$ by assumption and inference calculates that
its $n$-th pulse catches up to $B$ at a time $r_A(n)$ that satisfies the equation:
\begin{align}
v_B r_A(n) & = c(r_A(n) - nT) & & \text{$B$'s position equals light pulse's position} \nonumber \\
(v_B / c) r_A(n) & = r_A(n) - nT & & \text{divide by $c$} \nonumber \\
nT & = (1 - v_B/c) r_A(n) & & \text{a little more algebra} \nonumber \\
r_A(n) & = \frac{nT}{1 - v_B/c} \nonumber \\
r_A(n) & = \frac{nT}{1 - \beta_B} & & \text{Defn.\eqref{defn:betaB}} \label{eqn:scen1Arecvtime}
\end{align}
By the assumptions of special relativity, $A$ deduces that $B$'s clock
is running $\gamma_B$ times slower than $A$'s clock.
Thus $B$'s time when receiving the $n$-th pulse is:
\begin{align*}
r_B(n) & = r_A(n) / \gamma_B & & \text{$B$'s clock slower by factor $\gamma_B$} \\
r_B(n) & = \frac{nT}{\gamma_B(1 - \beta_B)} & & \text{Eqn.~\eqref{eqn:scen1Arecvtime}} \\
r_B(n) & = nT \frac{\sqrt{1 - \beta_B^2}}{1 - \beta_B} & & \text{Defn.~\eqref{defn:gammaB}} \\
r_B(n) & = nT \sqrt{ \frac{1 + \beta_B}{1 - \beta_B} } & & \text{a little algebra} \\
r_B(n) & = nT D_B & & \text{Defn.~\eqref{defn:DB}}
\end{align*}
$B$ will observe pulses arriving with period $D_B T$ according to
$B$'s clock.

Note from the previous section that $A$ sees this same period, on
$A$'s clock, for the period of pulses that $A$ receives from $B$.


\subsubsection{Relationship to Lorentz Ether Theory}
\label{sec:scen1LET}

Note in Sections~\ref{sec:scen1BtoA} and~\ref{sec:scen1AtoB}, that
except for algebra and the definitions of symbols from
Section~\ref{sec:sr}, the only assumptions we used from special relativity were:
\begin{itemize}
\item The one-way speed of light is $c$ in all directions, as measured
  by $A$.
\item $B$'s clock runs at a slower rate, by a factor of $1/\gamma_B$,
  relative to $A$'s clock.
\end{itemize}
By Lorentz Ether Theory, if $A$ is at rest relative to the ether, then:
\begin{itemize}
\item The one-way speed of light is $c$ in all directions relative to
  the ether, and thus $A$, at rest relative to the ether, will measure
  that speed for light in all directions.
\item $B$ is moving at velocity $v_B$ relative to the ether, and thus
  $B$'s clock physically runs at a slower rate, by a factor of
  $1/\gamma_B$.  $A$'s clock runs at the full rate of all clocks at
  that are at rest relative to the ether.
\end{itemize}

TODO: Find a way to explain the following better.

I had heard from learning about special relativity that when $A$ and
$B$ are moving at constant velocity towards or away from each other,
that $A$ observed that $B$'s clock ran slower by factor of $\gamma$,
and $B$ observed that $A$'s clock ran slower by a factor of $\gamma$.
(Note: I do not claim that those are fully precise statements, but
there is definitely a sense in which special relativity does say
something very much like this.)

In $A$'s frame observing $B$'s clock run slower, that seems perfectly
consistent with Lorentz Ether Theory's statement that if $A$ is at
rest relative to the ether, and $B$ moves at constant velocity $v_B$
relative to the ether, that $B$ experiences duration dilation,
i.e. its clock physically runs slower than $A$'s by a factor of
$\gamma_B$.  In this situation $A$'s clock runs at full speed,
i.e. $\gamma_B$ times {\em faster} than $B$'s.

But what about Lorentz Ether Theory's position on the converse
statement?  That is, from $B$'s point of view, does $B$ observe $A$'s
clock running $\gamma_B$ times slower?  If so, how can that possibly
make sense?

I now believe that the answer is that the statements in special
relativity can be made a bit more precise by saying something like
this: Because $A$ is following special relativity's assumptions,
i.e. in $A$'s frame the speed of light propagates isotropically at
constant speed $c$, therefore $A$ can deduce that any clocks moving at
constant speed $v$ directly towards or away from $A$ run $\gamma$
times slower, and make further calculations from that deduction.

$A$ does not actually {\em observe} such clocks directly over any
appreciable interval of time, so they are always, or almost always, so
far away that $A$ cannot make {\em any} direct observations of how
fast such clocks are running.  By ``direct'' observations I mean
``with light propagation delay very close to 0 between $A$ and the
entity being observed''.

Suppose in some future context of knowledge that not only is Lorentz
Ether Theory proven, but in such a way that we know how to measure our
speed relative to the ether.

Then, in the scanario described, we would know that $A$'s clock is
running at full speed, and light propagates isotropically at constant
speed $c$ relative to the ether, and thus also relative to $A$.

Everyone with this knowledge would be able to deduce that $B$'s clock
is running $\gamma$ times slower than full speed.  Also, that
$A$'s clock is running $\gamma$ times {\em faster} than $B$'s clock
(and that all of $A$'s local physical processes are proceeding
$\gamma$ times faster than similar local physical processes of $B$).

Further, light does {\em not} propagate at the same speed in all
directions relative to $B$.  It does so only with respect to the
ether.

We could also prove that if one chose to make calculations using
special relativity's assumptions in $B$'s frame, one would get the
same answers to these calculations that you do when using Lorentz
Ether Theory.

A hint of corroboration can be seen in the Wikipedia page on time
dilation, which says in the introduction~\cite{WikipediaTimeDilation}:
\begin{quote}
The dilation compares ``wristwatch'' clock readings between events
measured in different inertial frames and is not observed by visual
comparison of clocks across moving frames.
\end{quote}


\subsection{Special Relativity Scenario 2: Three entities, two of them moving away from the first at constant velocity}
\label{sec:scen2}

$A$ is at rest.
$B$ is moving at constant velocity $v_B$ away from $A$.
$C$ is moving at constant velocity $v_C > v_B$ away from $A$, in same
direction that $B$ is moving.

For $B$ sending periodic pulses to $A$ or vice versa, everything in
Sections~\ref{sec:scen1BtoA} and~\ref{sec:scen1AtoB} applies without
change.
For $C$ sending periodic pulses to $A$ or vice versa, everything in
Section~\ref{sec:scen1BtoA} and~\ref{sec:scen1AtoB} applies,
except replace $B$ subscripts
with $C$ subscripts, i.e. use $v_C$, $\gamma_C$, $\beta_C$, and $D_C$,
defined in the same way as for the versions with subscript $B$ in
Section~\ref{sec:sr}.

So it is only pulses between $B$ and $C$ that might present something
new here.


\subsubsection{$B$ sends periodic pulses to $C$}
\label{sec:scen2BtoCLETfriendly}

$B$ uses his local clock to time the sending of radio pulses to $C$,
sending pulses once every time interval $T$, according to $B$'s clock.
At what period does $C$ receive the pulses?

For simplicity of calculations, we will assume that both $B$ and $C$
pass $A$'s position at time 0, and $A$, $B$, and $C$ all synchronize
their clocks at that time.  As before, note that the calculation of
the period for someone receiving pulses is unaffected by whether such
synchronization is done.

In order to make the calculations as applicable to Lorentz Ether
Theory as possible, all calculations will be done in $A$'s inertial
frame.

Define $t_A(n)$ and $t_B(n)$ the same way as they were in
Section~\ref{sec:scen1BtoA}.  As explained there, when $A$ makes the
assumptions according to special relativity theory:
\begin{align}
t_B(n) & = nT & & \text{by the setup of the experiment} \nonumber \\
t_A(n) & = \gamma_B nT & & \text{$B$'s clock runs $\gamma_B$ times slower than $A$'s} \label{eqn:scen2timeA} \\
x_B(t_A) & = v_B t_A & & \text{relationship of $B$'s position and time, in $A$'s frame} \label{eqn:scen2posB} \\
x_C(t_A) & = v_C t_A & & \text{relationship of $C$'s position and time, in $A$'s frame}
\end{align}
Also by special relativity assumptions, $A$ considers the pulse to
travel from $B$ to $C$ at constant speed $c$.
The $n$-th pulse is emitted at time $t_A(n)$ in $A$'s frame,
so its position as a function of $A$'s time $t_A$ is:
\begin{align*}
l_A(n, t_A)
  & = \text{position of $B$ when emitted} \\
  &   + \text{distance traveled after emission} \\
  & = x_B(t_A(n)) + (t_A - t_A(n)) c & & \text{for any time $t_A \geq t_A(n)$} \\
  & = v_B t_A(n) + (t_A - t_A(n)) c & & \text{substitute Eqn.~\eqref{eqn:scen2posB}} \\
  & = -(c - v_B) t_A(n) + t_A c & & \text{algebra} \\
  & = -(c - v_B) \gamma_B nT + t_A c & & \text{substitute Eqn.~\eqref{eqn:scen2timeA}}
\end{align*}
To find $A$'s time when $C$ receives the pulse,
solve for $t_A$ that makes the pulse position the same as $C$'s position:
\begin{align*}
x_C(t_A) & = l_A(n, t_A) \\
v_C t_A  & = -(c - v_B) \gamma_B nT + t_A c \\
(c - v_B) \gamma_B nT & = (c - v_C) t_A & & \text{algebra} \\
t_A & = \frac{c - v_B}{c - v_C} \gamma_B nT & & \text{algebra} \\
t_A & = \frac{1 - \beta_B}{1 - \beta_C} \gamma_B nT & & \text{defn. of $\beta_B, \beta_C$}
\end{align*}
According to $A$ and its special relativity assumptions,
$C$'s clock runs slower, at a rate $1/\gamma_C$ times that of $A$'s clock.
So $C$'s time $r_C(n)$ to receive the $n$-th pulse sent by $B$ is:
\begin{align}
r_C(n) & = \frac{1}{\gamma_C} t_A \\
       & = \frac{\gamma_B}{\gamma_C} \left( \frac{1-\beta_B}{1-\beta_C} \right) nT \\
       & = \sqrt{ \frac{1-\beta_C^2}{1-\beta_B^2} } \left( \frac{1-\beta_B}{1-\beta_C} \right) nT \\
       & = \sqrt{ \frac{1+\beta_C}{1-\beta_C} } \sqrt { \frac{1-\beta_B}{1+\beta_B} } nT & & \text{algebra} \\
       & = ( D_C / D_B ) nT & & \text{defn. of $D_B, D_C$}
\end{align}
So when $B$ sends pulses with period $T$ according to $B$'s clock,
$C$ receives from $B$ pulses with period $(D_C / D_B) T$ on $C$'s clock.


\subsubsection{$B$ sends periodic pulses to $C$, double-check by ChatGPT}
\label{sec:scen2BtoCdoublecheck}

Since at the time of performing calculations in the previous section I
was still fairly new to such things, I wanted a way to double-check
the results.  I asked ChatGPT what the period would be that $C$ would
receive pulses from $B$ and it gave an answer close to the following.

Calculate the velocity of $B$ in $C$'s frame using relativistic
velocity subtraction:
\begin{align}
u' & = \frac{v_B - v_C}{1 - \frac{v_B v_C}{c^2}} \\
\beta' & = \frac{\beta_B - \beta_C}{1 - \beta_B \beta_C}
\end{align}

Since $v_B < v_C$, the numerator is negative.  The denominator is
positive.  Thus $u' < 0$ and $\beta' < 0$.  Thus in $C$'s frame, $B$
is moving in the negative $x$ direction, and $C$ is at rest.

For receding motion:
\begin{equation}
T_{\text{observed}} = T_{\text{emitted}} \sqrt{ \frac{1+\beta}{1-\beta} }
\end{equation}
where:
\begin{equation}
\beta = \frac{|u'|}{c} = |\beta'|
\end{equation}
Since $\beta' < 0$:
\begin{equation}
|\beta'| = -\beta' = \frac{\beta_C - \beta_B}{1 - \beta_B \beta_C}
\end{equation}
From here on this is mostly just algebra:
\begin{align*}
\frac{T_{\text{observed}}}{T_{\text{emitted}}}
  & = \sqrt{ \frac{1+\beta}{1-\beta} } \\
  & = \sqrt{ \frac{1 + \frac{\beta_C - \beta_B}{1 - \beta_B \beta_C}}{1 - \frac{\beta_C - \beta_B}{1 - \beta_B \beta_C}} } \\
  & = \sqrt{ \frac{1 - \beta_C \beta_B + (\beta_C - \beta_B)}{1 - \beta_C \beta_B - (\beta_C - \beta_B)} } \\
  & = \sqrt{ \frac{(1 + \beta_C) (1 - \beta_B)}{(1 - \beta_C) (1 + \beta_B)} } \\
  & = D_C / D_B & & \text{defn. of $D_B, D_C$}
\end{align*}
This is the same result, calculated in a fairly different way, using
the assumptions of special relativity, which ChatGPT is much better at
answering questions about than it is any alternatives that are not
special relativity.


\subsubsection{$C$ sends periodic pulses to $B$}
\label{sec:scen2CtoBLETfriendly}

The derivation is nearly identical to that in
Section~\ref{sec:scen2BtoCLETfriendly}.

The formula $l_A(n,t_A)$ for the position of the $n$-th pulse emitted
by $C$ at $A$'s time $t_A$ is:
\begin{equation}
l_A(n, t_A) = (c + v_C) \gamma_C nT - t_A c
\end{equation}
$A$'s time when $B$ receives the $n$-th pulse $t_A$ is:
\begin{equation}
t_A = \frac{1 + \beta_C}{1 + \beta_B} \gamma_C nT
\end{equation}
and $B$'s time when it receives the $n$-th pulse from $C$ is:
\begin{align*}
r_B(n) & = \frac{1}{\gamma_B} t_A \\
       & = ( D_C / D_B ) nT
\end{align*}
That is exactly the same local time period that $C$ measures for
pulses sent from $B$.


\subsubsection{A note on Doppler factors and relativistic velocity addition and subtraction}
\label{sec:dopplerfactorsandvelocity}

Note that a consequence of the derivation in
Section~\ref{sec:scen2BtoCdoublecheck} is the following.

If you do relativistic velocity subtraction of $u$ minus $v$,
resulting in $w$, then the Doppler factor of $w$ is $D_w = D_u / D_v$.

Although not shown in that section, it is pretty much the same algebra
to show that when you do relativistic velocity addition of $u$ plus
$v$ resulting in $w$, the Doppler factor of $w$ is $D_w = D_u D_v$.

This is apparently a well-known result among those working with
relativistic Doppler factors.


%\newpage
\bibliography{refs}

%\appendix

%\section{Omitted Proof in Section~\ref{sec:examples}}
%\label{app:1}

	
\end{document}
