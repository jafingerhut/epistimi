\documentclass[a4paper]{article}

\usepackage[pages=all, color=black, position={current page.south}, placement=bottom, scale=1, opacity=1, vshift=5mm]{background}
%\SetBgContents{
%	\tt This work is shared under a \href{https://creativecommons.org/licenses/by-sa/4.0/}{CC BY-SA 4.0 license} unless otherwise noted
%}      % copyright

\usepackage{physics}
\usepackage[margin=1in]{geometry} % full-width

% AMS Packages
\usepackage{amsmath}
\usepackage{amsthm}
\usepackage{amssymb}

% Unicode
\usepackage[utf8]{inputenc}
\usepackage{hyperref}
\hypersetup{
	unicode,
%	colorlinks,
%	breaklinks,
%	urlcolor=cyan, 
%	linkcolor=blue, 
	pdfauthor={Author One, Author Two, Author Three},
	pdftitle={A simple article template},
	pdfsubject={A simple article template},
	pdfkeywords={article, template, simple},
	pdfproducer={LaTeX},
	pdfcreator={pdflatex}
}

% Natbib
\usepackage[sort&compress,numbers,square]{natbib}
\bibliographystyle{mplainnat}

% Theorem, Lemma, etc
\theoremstyle{plain}
\newtheorem{theorem}{Theorem}
\newtheorem{corollary}[theorem]{Corollary}
\newtheorem{lemma}[theorem]{Lemma}
\newtheorem{claim}{Claim}[theorem]
\newtheorem{axiom}[theorem]{Axiom}
\newtheorem{conjecture}[theorem]{Conjecture}
\newtheorem{fact}[theorem]{Fact}
\newtheorem{hypothesis}[theorem]{Hypothesis}
\newtheorem{assumption}[theorem]{Assumption}
\newtheorem{proposition}[theorem]{Proposition}
\newtheorem{criterion}[theorem]{Criterion}
\theoremstyle{definition}
\newtheorem{definition}[theorem]{Definition}
\newtheorem{example}[theorem]{Example}
\newtheorem{remark}[theorem]{Remark}
\newtheorem{problem}[theorem]{Problem}
\newtheorem{principle}[theorem]{Principle}

\usepackage{graphicx, color}
\graphicspath{{fig/}}

%\usepackage[linesnumbered,ruled,vlined,commentsnumbered]{algorithm2e} % use algorithm2e for typesetting algorithms
\usepackage{algorithm, algpseudocode} % use algorithm and algorithmicx for typesetting algorithms
\usepackage{mathrsfs} % for \mathscr command

% Author info
\title{Andy's math/science background information}
\author{J. Andrew Fingerhut (\texttt{andy.fingerhut@gmail.com})}

\date{
        June 1, 2025
%	\today
}

\newcommand{\ihat}{\textbf{i}}
\newcommand{\jhat}{\textbf{j}}
\newcommand{\khat}{\textbf{k}}
\newcommand{\vect}[1]{\textbf{#1}}
\newcommand{\hatvec}[1]{\hat{\textbf{#1}}}

\newcommand{\reals}{\mathbb{R}}
\newcommand{\del}{\nabla}
%\newcommand{\grad}{grad}

\begin{document}
\maketitle

%\begin{abstract}
%  todo: abstract here
%\end{abstract}

\tableofcontents

\section{Introduction}
\label{sec:intro}

This document is a place to write up notes on background in math that
help in learning science.


\section{Notation}
\label{sec:notation}

$\reals^3$ - The set of all points in 3-dimensional space,
i.e. where each point is specified by a sequence of 3 real-valued
coordinates.

$\del$ - Called ``nabla'', and often called ``del''.  Some history of
it can be found on the Nabla Symbol page~\cite{NablaSymbol}.

$\grad f$ - Gradient of a scalar function $f: \reals^3 \rightarrow
\reals$.

$\text{grad} f$ - another way to write the gradient of $f$, $\grad f$

TODO: Any 3Blue1Brown-quality YouTube videos showing examples and
definition of gradient, divergence, and curl?  If so, give links to
them below in appropriate sections.


\subsection{Maxwell's Equations}
\label{sec:maxwellseqns}

In partial differentiatal form:

\begin{align*}
  \grad \cdot \vect{E} & = \frac{\rho}{\epsilon_0} & & \text{Gauss's Law} \\
  \grad \cdot \vect{B} & = 0 & & \text{Gauss's Law for magnetism} \\
  \grad \times \vect{E} & = - \frac{\partial \vect{B}}{\partial t} & & \text{Maxwell-Faraday Equation (Faraday's law of induction)} \\
  \grad \times \vect{B} & = \mu_0 ( \vect{J} + \epsilon_0 \frac{\partial \vect{E}}{\partial t} ) & & \text{Amp\`{e}re-Maxwell law}
\end{align*}
where:
\begin{itemize}
  \item $\vect{E}$ is the electric field
  \item $\vect{B}$ is the magnetic field
  \item $\rho$ is the electric charge density
  \item $\vect{J}$ is the current density
  \item $\epsilon_0$ is the vacuum permittivity
  \item $\mu_0$ is the vacuum permeability
\end{itemize}


\section{Gradient}
\label{sec:gradient}

Gradient has been generalized to many coordinate systems other than
the 3-dimensional Cartesian coordinate system $\mathbb{R}^3$, but I
will focus on $\mathbb{R}^3$.  The Wikipedia page on
Gradient~\cite{Gradient} is not too bad for me, as long as I skim over
the parts that generalize it to other coordinate systems.

Griffiths~\cite{Griffiths1998} Section 1.2.2 ``Gradient'' is good at
giving the definition and some useful examples and properties of the
gradient.  He defines $\grad T$ this way, where $\hatvec{x}$,
$\hatvec{y}$, and $\hatvec{z}$ are the unit vectors in the direction
of the three coordinate axes:
\begin{align*}
  \grad T & \equiv \frac{\partial T}{\partial x} \hatvec{x}
                 + \frac{\partial T}{\partial y} \hatvec{y}
                 + \frac{\partial T}{\partial z} \hatvec{z}
\end{align*}
where:
\begin{itemize}
  \item $T : \reals^3 \rightarrow \reals$ is a scalar function that is
    continuous and differentiable
\end{itemize}
Some properties of the gradient:
\begin{itemize}
  \item $\grad T(x, y, z)$ is the gradient evaluated at a position
    given by $(x, y, z)$.  The vector points in a direction that
    function $T$ increases most quickly.
  \item The function $T$ can be approximated at points $\vect{r}$ near
    $\vect{r}_0 = (x, y, z)$ by the linear function $T(\vect{r}_0) +
    (\vect{r} - \vect{r}_0) \cdot \grad T(\vect{r}_0)$
  \item The instantaneous rate of change of $T$ in direction
    $\vect{u}$ (a unit vector) from $\vect{r}_0$ is $\vect{u} \cdot
    \grad T(\vect{r}_0)$.  Note that it is always 0 in a direction
    perpendicular to $\grad T(\vect{r}_0)$, and the negative of the
    magnitude of $\grad T(\vect{r}_0)$ in the opposite direction.
  \item Consider a ``level set'', i.e. a surface defined by all of the
    points $\vect{r}$ where $T(\vect{r})=c$ for some constant $c$.
    Then $\grad T$ evaluated at any point on that surface, is normal
    to the surface.
\end{itemize}


%\newpage
\bibliography{refs}

%\appendix

%\section{Omitted Proof in Section~\ref{sec:examples}}
%\label{app:1}

	
\end{document}
