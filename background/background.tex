\documentclass[a4paper]{article}

\usepackage[pages=all, color=black, position={current page.south}, placement=bottom, scale=1, opacity=1, vshift=5mm]{background}
%\SetBgContents{
%	\tt This work is shared under a \href{https://creativecommons.org/licenses/by-sa/4.0/}{CC BY-SA 4.0 license} unless otherwise noted
%}      % copyright

\usepackage{physics}
\usepackage[margin=1in]{geometry} % full-width

% AMS Packages
\usepackage{amsmath}
\usepackage{amsthm}
\usepackage{amssymb}

% Unicode
\usepackage[utf8]{inputenc}
\usepackage{hyperref}
\hypersetup{
	unicode,
%	colorlinks,
%	breaklinks,
%	urlcolor=cyan, 
%	linkcolor=blue, 
	pdfauthor={Author One, Author Two, Author Three},
	pdftitle={A simple article template},
	pdfsubject={A simple article template},
	pdfkeywords={article, template, simple},
	pdfproducer={LaTeX},
	pdfcreator={pdflatex}
}

% Natbib
\usepackage[sort&compress,numbers,square]{natbib}
\bibliographystyle{mplainnat}

% Theorem, Lemma, etc
\theoremstyle{plain}
\newtheorem{theorem}{Theorem}
\newtheorem{corollary}[theorem]{Corollary}
\newtheorem{lemma}[theorem]{Lemma}
\newtheorem{claim}{Claim}[theorem]
\newtheorem{axiom}[theorem]{Axiom}
\newtheorem{conjecture}[theorem]{Conjecture}
\newtheorem{fact}[theorem]{Fact}
\newtheorem{hypothesis}[theorem]{Hypothesis}
\newtheorem{assumption}[theorem]{Assumption}
\newtheorem{proposition}[theorem]{Proposition}
\newtheorem{criterion}[theorem]{Criterion}
\theoremstyle{definition}
\newtheorem{definition}[theorem]{Definition}
\newtheorem{example}[theorem]{Example}
\newtheorem{remark}[theorem]{Remark}
\newtheorem{problem}[theorem]{Problem}
\newtheorem{principle}[theorem]{Principle}

\usepackage{graphicx, color}
\graphicspath{{fig/}}

%\usepackage[linesnumbered,ruled,vlined,commentsnumbered]{algorithm2e} % use algorithm2e for typesetting algorithms
\usepackage{algorithm, algpseudocode} % use algorithm and algorithmicx for typesetting algorithms
\usepackage{mathrsfs} % for \mathscr command

% Author info
\title{Andy's math/science background information}
\author{J. Andrew Fingerhut (\texttt{andy.fingerhut@gmail.com})}

\date{
        June 1, 2025
%	\today
}

\newcommand{\ihat}{\textbf{i}}
\newcommand{\jhat}{\textbf{j}}
\newcommand{\khat}{\textbf{k}}
\newcommand{\vect}[1]{\textbf{#1}}
\newcommand{\hatvec}[1]{\hat{\textbf{#1}}}

\newcommand{\reals}{\mathbb{R}}
\newcommand{\del}{\nabla}
%\newcommand{\grad}{grad}

\begin{document}
\maketitle

%\begin{abstract}
%  todo: abstract here
%\end{abstract}

\tableofcontents

\section{Introduction}
\label{sec:intro}

This document is a place to write up notes on background in math that
help in learning science.


\section{Notation}
\label{sec:notation}

$\reals^3$ - The set of all points in 3-dimensional space,
i.e. where each point is specified by a sequence of 3 real-valued
coordinates.

$\del$ - Called ``nabla'', and often called ``del''.  Some history of
it can be found on the Nabla Symbol page~\cite{NablaSymbol}.

$\grad f$ - Gradient of a scalar function $f: \reals^3 \rightarrow
\reals$.

$\text{grad} f$ - another way to write the gradient of $f$, $\grad f$

TODO: Any 3Blue1Brown-quality YouTube videos showing examples and
definition of gradient, divergence, and curl?  If so, give links to
them below in appropriate sections.


\subsection{Maxwell's Equations}
\label{sec:maxwellseqns}

In partial differentiatal form:

\begin{align}
  \grad \cdot \vect{E} & = \frac{\rho}{\epsilon_0} & & \text{Gauss's Law} \label{eq:gaussE} \\
  \grad \cdot \vect{B} & = 0 & & \text{Gauss's Law for magnetism} \label{eq:gaussB} \\
  \grad \times \vect{E} & = - \frac{\partial \vect{B}}{\partial t} & & \text{Maxwell-Faraday Equation (Faraday's law of induction)} \label{eq:maxfar} \\
  \grad \times \vect{B} & = \mu_0 ( \vect{J} + \epsilon_0 \frac{\partial \vect{E}}{\partial t} ) & & \text{Amp\`{e}re-Maxwell law} \label{eq:ampmax}
\end{align}
where:
\begin{itemize}
  \item $\vect{E}$ is the electric field
  \item $\vect{B}$ is the magnetic field
  \item $\rho$ is the electric charge density
  \item $\vect{J}$ is the current density
  \item $\epsilon_0$ is the vacuum permittivity
  \item $\mu_0$ is the vacuum permeability
\end{itemize}

Statement of the equations in English:

Gauss's Law (equation~\eqref{eq:gaussE}) says that electric fields are
``produced'' radially outward from volumes with positive electric
charge, and ``consumed'' radially into volumes with negative electric
charge.

Gauss's Law for Magnetism (equation~\eqref{eq:guassB}) says that there
are no magnetic monopoles, i.e. no sources or sinks for net magnetic
field.  All magnetic flux ``swirls around'' in loops.

\subsubsection{Questions about Maxwell's Equations}

\paragraph{Questions about Gauss's Law}

TODO: Is Gauss's Law a strict generalization of Coulomb's Law?  In
what ways?  One way is that Gauss's Law gives the electric field
generated by a distribution of charge, whereas Coulomb's Law gives the
force between two point charges.  So applying to a continuous
distribution of charge is one generalization, and what electric field
is created by that distribution of charge is another (as long as you
then combine Gauss's Law with the Lorentz force law in order to
calculate the forces).

TODO: It seems that Gauss's Law, if you take $\rho$ and $\vect{E}$ as
functions of time, gives an electric field that changes
instantaneously everywhere when the charge distribution $\rho$
changes.  There does not appear to be anything related to retarded
positions of charge in that equation anywhere.  How does the finite
speed of electric field propagation enter into Maxwell's equations?

Doing a Google search on the question ``is Gauss's law consistent with
relativistic effects'' gives answers that it {\em is} consistent with
special relativity, including the following paragraphs:

\begin{itemize}
\item
Maxwell's equations are relativistic: Gauss's law is one of Maxwell's
equations.  These equations, which describe classical
electromagnetism, are inherently Lorentz invariant.  This means they
hold true in all inertial reference frames, consistent with the
principle of relativity, which states that the laws of physics are the
same for all observers in uniform motion relative to one another.
\item
Lorentz covariance: Maxwell's equations, and thus Gauss's law, can be
written in a covariant form that is explicitly consistent with Lorentz
transformations.  These transformations describe how physical
quantities change when moving from one inertial frame to another,
incorporating relativistic effects like time dilation and length
contraction.
\end{itemize}


\paragraph{Questions about Gauss's Law for Magnetism}

Figures for electromagnetic waves that are the propagation of light in
a vacuum show E and B fields as perpendicular to each other, and each
a sinusoid with the same phase as each other.  These figures must be
dramatic simplifications of the full E and B fields, e.g. perhaps they
show at one instant in time all E and B fields only along the line
that is the ``center'' of the ray of light?  In particular, I am
wondering how they look in all of space, such that it is clear that
their divergence is 0 at all times, at all positions in space.  It
might be instructive to have a visual representation, perhaps an
animation, somehow showing what the E field is at all points in space
at one particular instant of time.  The B field has a very similar
shape.  And if you know what the shape of these fields are at one
instant of time, then you know that they merely ``shift'' in the
direction of light propagation over time at speed $c$, without
changing shape.


\paragraph{Is there a way to write them that has only $\rho$ without $J$?}

If one emphasizes the view that all current is due to moving charge,
is there a way to write Maxwell's equations in a way that makes this
clear?  For example, is there a way to write it with some single
function $x$ that somehow represents both $\rho$ and $J$?

I realize that it is possible to have non-zero $J$ even when $\rho$ is
a constant over time, e.g. a wire carrying a constant current in a
loop.  Here charge is moving, but the density of the charge at any one
point in space remains the same over time, because in any given
sub-volume within the wire, as much charge is entering that volume per
unit time as is leaving per unit time.

Answer from Google to this search phrase: ``are there versions of
Maxwell's equations that combine charge density and current density''

Yes, Maxwell's equations can be formulated in a way that combines
charge density and current density into a single entity called the
four-current.  This formulation is part of the covariant formulation
of electromagnetism, which is a more elegant and relativistic way to
express Maxwell's equations.

More details available on the Wikipedia page ``Covariant formulation
of classical
electromagnetism''~\cite{CovariantFormulationOfClassicalElectromagnetism}.

Since this is a four-vector, it has just as much ``information'' as
the scalar field $\rho$ plus the vector field $J$.  It makes sense
that if someone had found a way to combine $\rho$ and $J$ into a
single time-varying scalar field, they would have done so long ago.


\section{Gradient}
\label{sec:gradient}

Gradient has been generalized to many coordinate systems other than
the 3-dimensional Cartesian coordinate system $\mathbb{R}^3$, but I
will focus on $\mathbb{R}^3$.  The Wikipedia page on
Gradient~\cite{Gradient} is not too bad for me, as long as I skim over
the parts that generalize it to other coordinate systems.

Griffiths~\cite{Griffiths1998} Section 1.2.2 ``Gradient'' is good at
giving the definition and some useful examples and properties of the
gradient.  He defines $\grad T$ this way, where $\hatvec{x}$,
$\hatvec{y}$, and $\hatvec{z}$ are the unit vectors in the direction
of the three coordinate axes:
\begin{align*}
  \grad T & \equiv \frac{\partial T}{\partial x} \hatvec{x}
                 + \frac{\partial T}{\partial y} \hatvec{y}
                 + \frac{\partial T}{\partial z} \hatvec{z}
\end{align*}
where:
\begin{itemize}
  \item $T : \reals^3 \rightarrow \reals$ is a scalar function that is
    continuous and differentiable
\end{itemize}
Some properties of the gradient:
\begin{itemize}
  \item $\grad T(x, y, z)$ is the gradient evaluated at a position
    given by $(x, y, z)$.  The vector points in a direction that
    function $T$ increases most quickly.
  \item The function $T$ can be approximated at points $\vect{r}$ near
    $\vect{r}_0 = (x, y, z)$ by the linear function $T(\vect{r}_0) +
    (\vect{r} - \vect{r}_0) \cdot \grad T(\vect{r}_0)$
  \item The instantaneous rate of change of $T$ in direction
    $\vect{u}$ (a unit vector) from $\vect{r}_0$ is $\vect{u} \cdot
    \grad T(\vect{r}_0)$.  Note that it is always 0 in a direction
    perpendicular to $\grad T(\vect{r}_0)$, and the negative of the
    magnitude of $\grad T(\vect{r}_0)$ in the opposite direction.
  \item Consider a ``level set'', i.e. a surface defined by all of the
    points $\vect{r}$ where $T(\vect{r})=c$ for some constant $c$.
    Then $\grad T$ evaluated at any point on that surface, is normal
    to the surface.
\end{itemize}


\section{Galilean relativity}

In classical Newtonian mechanics, physical laws are invariant under
Galilean relativity, i.e. any laws that hold in one inertial frame
also hold in all other inertial frames, i.e. those that move with a
constant velocity relative to the first frame.

This applies to Newton's three laws of motion, for example.

It does not apply to measurements of quantities that depend upon
velocity, e.g. an object's momentum or kinetic energy.

However, the properties that momentum is conserved, and energy is
conserved, is preserved across inertial frames, even if the absolute
values of the kinetic energy and momentum differ between frames.

If there are forces that are velocity-dependent, e.g. a frictional
force resisting an object traveling through a medium like air or water
that is a function of velocity relative to that medium, then as long
as that force is always written as a function of the relative velocity
between the object and the medium, then note that the measurements of
such relative velocities are the same in different inertial frames.

For other examples and some mathematical derivations,
see~\cite{GalileanTransformation}.

See this video~\cite{GalileanInvarianceMaxwellsEquations} that gives a
similar mathematical demonstration that Newton's second law is
invariant under Galilean relativity, but Maxwell's equations are {\em
  not}.


\section{Experiments}

\begin{itemize}
\item 1848-49: Fizeau's measurement of the speed of light in
  air~\cite{Fizeau1849}
\item 1851: Fizeau's experiment to measure the relative speed of light
  in a moving medium (water)~\cite{Fizeau1851}
\end{itemize}


\section{Questions}

\begin{itemize}
\item Historically, how was aberration of light
  distinguished/separated from other effects like parallax?
\end{itemize}


\section{Index of A History of the Theories of Aether and Electricity}

All page numbers are of the form ``I(number)'' for a page number in
Volume I, or ``II(number)'' for a page in Volume II.

Done for I, pp. 1-73

\begin{itemize}
\item aberration of light: I22
\item action at a distance: I5, I28, I30-31, I36, I48, I50, I51, I59
\item Aepinus, Franz Ulrich: I51
\item aether: I5, I13, I19, I20, I30
\item atomism: I13
\item Bartholin, Erasmus: I25
\item birefringence, discovered 1669: I25
\item Bologna stone: I20
\item Cavendish, Henry: I54
\item Cavendish, Charles: I54
\item colors of light, Descartes' theory: I9
\item colors of light, Hooke's theory: I16
\item colors of light, Newton's theory: I17
\item colors of thin plates: I13, I21
\item conductor, origin of name: I42
\item corpuscular theory of light: I10, I20, I23, I31
\item Coulomb, Augustin: I57
\item de Maricourt: see entry alphabetized as ``Maricourt''
\item Desaguliers, Jean Th\'{e}ophile: I41
\item diamond emitting light when rubbed or heated: I14
\item dielectric: I52
\item diffraction: I14, I20
\item du Fay: see entry alphabetized as ``Fay''
\item epistemological rationalism: I7
\item electric charge, conservation of: I47
\item electric charge, induction of: I52
\item electric current through human body, used as sensory evidence: I45, I55, I67, I71, I73
\item electric fluid: I42
\item electric fluid, one-fluid theory: I50
\item electric fluid, two-fluid theory: I58
\item electric force, how it varies with distance: I53, I54, I57
\item electric fluid in conductor is all at surface at equilibrium: I51, I53, I61
\item electric pile: I71-73
\item electrolysis of water: I74
\item electrical conducting power of different materials and shaped: I55
\item electrification, discovery of two kinds by du Fay in 1733: I44
\item electrification only at the surface, discovery in 1729: I43, I49
\item electrical emanations: I36
\item electricity, discoveries by Gilbert, and naming: I35
\item electricity, discovery of transfer between electrified bodies: I42
\item electricity, ordinary matter (without electric fluid) repels other ordinary matter: I52
\item electricity, resinous: I44
\item electricity, theory of electric fluid that electrified bodies have a surplus or deficiency of when electrified: I46, I49
\item electricity, vitreous: I44
\item electrostatic mathematics by Poisson: I61-62
\item extraordinary ray: see birefringence
\item du Fay, Charles-Fran\c{c}ois: I43
\item fits of easy transmission and easy reflection, theory of, about light: I22
\item Franklin, Benjamin: I45, I47
\item frogs in electrical experiments: I67-69
\item Galvani, Luigi: I67
\item galvanism, i.e. electric current caused by touching of different metals: I70
\item galvanism, proof tha it was electrical in nature: I71
\item Gilbert, William: I34
\item Gravesande: see entry alphabetized as ``'s Gravesande''
\item Gray, Stephen: I41
\item Green, George: I65
\item Grene's theorem: I65
\item gravity: I28, I31
\item heat: I20, I38
\item heat, from focused light: I23
\item heat vs. light, heat absorbed by glass but not light: I40
\item Huygens' principle of light wave propagation: I24
\item inducation of electric charge: see electric charge, induction of
\item inter-phenomena: I30
\item Leyden phial, Leyden jar: I45
\item magnet, use as a compass, as early as 1186 in Europe: I33
\item magnet, Earth is a magnet: I34
\item magnetic fluid, two-fluid theory: I57, I59
\item magnetic force between poles decreases as square of distance: I56, I59
\item magnetic moment: I63
\item magnetic poles, discovery and naming: I33
\item magnetism, discoveries by Gilbert: I35
\item magnetism, distinguishing behavior from gravity: I55
\item de Maricourt, Pierre, also known as Peregrinus: I33
\item mechanical philosophy: I6
\item Michell, John: I56
\item van Musschenbroek, Pieter: I45
\item Neckam, Alexander: I33
\item Newton's rings: I13
\item opacity: I25
\item plenum: I5, I13
\item Poisson, Sim\'{e}on Denis: I60
\item polarized light: I19, I27
\item poles of magnets: see magnetic poles
\item potential: I65
\item Priestley, Joseph: I53
\item principle of least time: I12
\item reflection of light: I10, I24
\item refraction of light: I10, I15, I17, I20, I25
\item resinous: see electricity, resinous
\item Roemer, Olaf: I22
\item 's Gravesande, Wilhelm Jacob: I37
\item Snell, Willebrord: I10
\item sound as periodic vibrational waves: I21
\item speed of light propagation: I22
\item Torricelli: I24
\item wave theory of light: I14, I23
\item Wilcke, Johan Carl: I51
\item vacuum, light traveling through: I24
\item van Musschenbroek: see entry alphabetized as ``Musschenbroek''
\item vitreous: see electricity, vitreous
\item void: I5, I13
\item Volta, Alessandro: I69
\item Watson, William: I45
\end{itemize}

\subsection{Noteworthy events in aether vs. action at a distance}

Vol. I, p. 28:

\begin{quote}
  Newton claimed nothing more for his discovery than that it provided
  the necessary instrument for methematical prediction, and he pointed out
  that it did not touch on the question of the mechanism of gravity.
  As to this, he conjectured that the density of the aether might vary from
  place to place, and that bodies might tend to move from the denser parts
  of the medium toward the rarer; but whether this were the true explanation
  or not, at any rate, he said, to suppose 'that one body may act upon
  another at a distance through a vacuum, without the mediation of anything
  else, ... is to me so great an absurdity, that I believe no man, who has
  in philosophical matters a competent faculty for thinking, can ever fall
  into.'
\end{quote}

According to Wikipedia~\cite{WikipediaNewtonLawOfGravity}, Newton
wrote this in 1692 in a letter to Richard Bentley, 5 years after
Principia was first published in 1687.  I do not know if Newton said
anything similar in Principia itself.  Very likely he did not, or else
it would be more widely quoted on this topic.

Vol. I, p. 30:

\begin{quote}
  The rejection of the inverse-square law of gravitation by the French
  Cartesians antagonised the younger disciples of Newton to such an
  extent that the latter hardened into opposition not only to the
  vortices but to the whole body of Cartesian notions, including the
  aether.  In the second edition (1713) of the {\em Principia}, there
  is a preface written by Roger Cotes (1682-1716), in which the
  Newtonian law of action at a distance is championed as being the
  only formulation of the facts of experience which does not introduce
  unverifiable and useless suppositions.
\end{quote}

Vol. I, p. 31:

\begin{quote}
  When the eighteenth-century natural philosophers found by experience
  that the Newtonian law was marvellously powerful, yielding formulae
  by which practically every observable motion in the solar system
  could be predicted, while on the other hand the search for an
  explanation of inter-phenomena led to no practical results, opinion
  set in favour of Cotes's attitude, which came to prevail widely; and
  in the middle of the century R. G. Boscovich (1711-87) ... attempted
  to account for all known physical effects in terms of action at a
  distance between point particles.
\end{quote}

%\newpage
\bibliography{refs}

%\appendix

%\section{Omitted Proof in Section~\ref{sec:examples}}
%\label{app:1}

	
\end{document}
