\documentclass[a4paper]{article}

\usepackage[pages=all, color=black, position={current page.south}, placement=bottom, scale=1, opacity=1, vshift=5mm]{background}
%\SetBgContents{
%	\tt This work is shared under a \href{https://creativecommons.org/licenses/by-sa/4.0/}{CC BY-SA 4.0 license} unless otherwise noted
%}      % copyright

\usepackage{physics}
\usepackage[margin=1in]{geometry} % full-width

% AMS Packages
\usepackage{amsmath}
\usepackage{amsthm}
\usepackage{amssymb}
\usepackage{bm}

% Unicode
\usepackage[utf8]{inputenc}
\usepackage{hyperref}
\hypersetup{
	unicode,
%	colorlinks,
%	breaklinks,
%	urlcolor=cyan, 
%	linkcolor=blue, 
	pdfauthor={Author One, Author Two, Author Three},
	pdftitle={A simple article template},
	pdfsubject={A simple article template},
	pdfkeywords={article, template, simple},
	pdfproducer={LaTeX},
	pdfcreator={pdflatex}
}

% Natbib
\usepackage[sort&compress,numbers,square]{natbib}
\bibliographystyle{mplainnat}

% Theorem, Lemma, etc
\theoremstyle{plain}
\newtheorem{theorem}{Theorem}
\newtheorem{corollary}[theorem]{Corollary}
\newtheorem{lemma}[theorem]{Lemma}
\newtheorem{claim}{Claim}[theorem]
\newtheorem{axiom}[theorem]{Axiom}
\newtheorem{conjecture}[theorem]{Conjecture}
\newtheorem{fact}[theorem]{Fact}
\newtheorem{hypothesis}[theorem]{Hypothesis}
\newtheorem{assumption}[theorem]{Assumption}
\newtheorem{proposition}[theorem]{Proposition}
\newtheorem{criterion}[theorem]{Criterion}
\theoremstyle{definition}
\newtheorem{definition}[theorem]{Definition}
\newtheorem{example}[theorem]{Example}
\newtheorem{remark}[theorem]{Remark}
\newtheorem{problem}[theorem]{Problem}
\newtheorem{principle}[theorem]{Principle}

\usepackage{graphicx, color}
\graphicspath{{fig/}}

%\usepackage[linesnumbered,ruled,vlined,commentsnumbered]{algorithm2e} % use algorithm2e for typesetting algorithms
\usepackage{algorithm, algpseudocode} % use algorithm and algorithmicx for typesetting algorithms
\usepackage{mathrsfs} % for \mathscr command

% Author info
\title{Andy's math/science background information}
\author{J. Andrew Fingerhut (\texttt{andy.fingerhut@gmail.com})}

\date{
        June 1, 2025
%	\today
}

\newcommand{\ihat}{\textbf{i}}
\newcommand{\jhat}{\textbf{j}}
\newcommand{\khat}{\textbf{k}}
\newcommand{\vect}[1]{\textbf{#1}}
\newcommand{\hatvec}[1]{\hat{\textbf{#1}}}

\newcommand{\reals}{\mathbb{R}}
\newcommand{\del}{\bm{\nabla}}
%\newcommand{\grad}{grad}

\begin{document}
\maketitle

%\begin{abstract}
%  abstract here
%\end{abstract}

\tableofcontents

\section{Introduction}
\label{sec:intro}

This document is a place to write up notes on background in math that
help in learning science.


\section{Notation}
\label{sec:notation}

$\reals^3$ - The set of all points in 3-dimensional space,
i.e. where each point is specified by a sequence of 3 real-valued
coordinates.

$\del$ - Called ``nabla'', and often called ``del''.  Some history of
it can be found on the Nabla Symbol page~\cite{NablaSymbol}.

$\del f$ - Gradient of a scalar function $f: \reals^3 \rightarrow
\reals$.

$\text{grad} f$ - another way to write the gradient of $f$, $\del f$

TODO: Any 3Blue1Brown-quality YouTube videos showing examples and
definition of gradient, divergence, and curl?  If so, give links to
them below in appropriate sections.


\subsection{Maxwell's Equations}
\label{sec:maxwellseqns}

In partial differentiatal form:

\begin{align}
  \del \cdot \vect{E} & = \frac{\rho}{\epsilon_0} & & \text{Gauss's Law} \label{eq:gaussE} \\
  \del \cdot \vect{B} & = 0 & & \text{Gauss's Law for magnetism} \label{eq:gaussB} \\
  \del \times \vect{E} & = - \frac{\partial \vect{B}}{\partial t} & & \text{Maxwell-Faraday Equation (Faraday's law of induction)} \label{eq:maxfar} \\
  \del \times \vect{B} & = \mu_0 ( \vect{J} + \epsilon_0 \frac{\partial \vect{E}}{\partial t} ) & & \text{Amp\`{e}re-Maxwell law} \label{eq:ampmax}
\end{align}
where:
\begin{itemize}
  \item $\vect{E}$ is the electric field
  \item $\vect{B}$ is the magnetic field
  \item $\rho$ is the electric charge density
  \item $\vect{J}$ is the current density
  \item $\epsilon_0$ is the vacuum permittivity
  \item $\mu_0$ is the vacuum permeability
\end{itemize}

Statement of the equations in English:

Gauss's Law (equation~\eqref{eq:gaussE}) says that electric fields are
``produced'' radially outward from volumes with positive electric
charge, and ``consumed'' radially into volumes with negative electric
charge.

Gauss's Law for Magnetism (equation~\eqref{eq:guassB}) says that there
are no magnetic monopoles, i.e. no sources or sinks for net magnetic
field.  All magnetic flux ``swirls around'' in loops.

\subsubsection{Questions about Maxwell's Equations}

\paragraph{Questions about Gauss's Law}

TODO: Is Gauss's Law a strict generalization of Coulomb's Law?  In
what ways?  One way is that Gauss's Law gives the electric field
generated by a distribution of charge, whereas Coulomb's Law gives the
force between two point charges.  So applying to a continuous
distribution of charge is one generalization, and what electric field
is created by that distribution of charge is another (as long as you
then combine Gauss's Law with the Lorentz force law in order to
calculate the forces).

TODO: It seems that Gauss's Law, if you take $\rho$ and $\vect{E}$ as
functions of time, gives an electric field that changes
instantaneously everywhere when the charge distribution $\rho$
changes.  There does not appear to be anything related to retarded
positions of charge in that equation anywhere.  How does the finite
speed of electric field propagation enter into Maxwell's equations?

Doing a Google search on the question ``is Gauss's law consistent with
relativistic effects'' gives answers that it {\em is} consistent with
special relativity, including the following paragraphs:

\begin{itemize}
\item
Maxwell's equations are relativistic: Gauss's law is one of Maxwell's
equations.  These equations, which describe classical
electromagnetism, are inherently Lorentz invariant.  This means they
hold true in all inertial reference frames, consistent with the
principle of relativity, which states that the laws of physics are the
same for all observers in uniform motion relative to one another.
\item
Lorentz covariance: Maxwell's equations, and thus Gauss's law, can be
written in a covariant form that is explicitly consistent with Lorentz
transformations.  These transformations describe how physical
quantities change when moving from one inertial frame to another,
incorporating relativistic effects like time dilation and length
contraction.
\end{itemize}


\paragraph{Questions about Gauss's Law for Magnetism}

Figures for electromagnetic waves that are the propagation of light in
a vacuum show E and B fields as perpendicular to each other, and each
a sinusoid with the same phase as each other.  These figures must be
dramatic simplifications of the full E and B fields, e.g. perhaps they
show at one instant in time all E and B fields only along the line
that is the ``center'' of the ray of light?  In particular, I am
wondering how they look in all of space, such that it is clear that
their divergence is 0 at all times, at all positions in space.  It
might be instructive to have a visual representation, perhaps an
animation, somehow showing what the E field is at all points in space
at one particular instant of time.  The B field has a very similar
shape.  And if you know what the shape of these fields are at one
instant of time, then you know that they merely ``shift'' in the
direction of light propagation over time at speed $c$, without
changing shape.


\paragraph{Is there a way to write them that has only $\rho$ without $J$?}

If one emphasizes the view that all current is due to moving charge,
is there a way to write Maxwell's equations in a way that makes this
clear?  For example, is there a way to write it with some single
function $x$ that somehow represents both $\rho$ and $J$?

I realize that it is possible to have non-zero $J$ even when $\rho$ is
a constant over time, e.g. a wire carrying a constant current in a
loop.  Here charge is moving, but the density of the charge at any one
point in space remains the same over time, because in any given
sub-volume within the wire, as much charge is entering that volume per
unit time as is leaving per unit time.

Answer from Google to this search phrase: ``are there versions of
Maxwell's equations that combine charge density and current density''

Yes, Maxwell's equations can be formulated in a way that combines
charge density and current density into a single entity called the
four-current.  This formulation is part of the covariant formulation
of electromagnetism, which is a more elegant and relativistic way to
express Maxwell's equations.

More details available on the Wikipedia page ``Covariant formulation
of classical
electromagnetism''~\cite{CovariantFormulationOfClassicalElectromagnetism}.

Since this is a four-vector, it has just as much ``information'' as
the scalar field $\rho$ plus the vector field $J$.  It makes sense
that if someone had found a way to combine $\rho$ and $J$ into a
single time-varying scalar field, they would have done so long ago.


\paragraph{Diffraction}

Things that I have heard affect the results:

\begin{itemize}
  \item Wavelength(s) of the light involved.
  \item If there are slits involved, the number, width, and separation
    of the slits, and whether the source of light is ``coherent''.
  \item If there are not slits involved, the width and shape of the
    material blocking some of the light from the source.
  \item Does polarization of the light ever affect the results?
\end{itemize}


\section{Gradient}
\label{sec:gradient}

Gradient has been generalized to many coordinate systems other than
the 3-dimensional Cartesian coordinate system $\reals^3$, but I will
focus on $\reals^3$.  The Wikipedia page on Gradient~\cite{Gradient}
is fairly clear for me, as long as I skim over the parts that
generalize it to other coordinate systems.

Griffiths~\cite{Griffiths1998} Section 1.2.2 ``Gradient'' is good at
giving the definition and some useful examples and properties of the
gradient.  He defines $\del T$ this way, where $\hatvec{x}$,
$\hatvec{y}$, and $\hatvec{z}$ are the unit vectors in the direction
of the three coordinate axes:
\begin{align*}
  \del T & \equiv \frac{\partial T}{\partial x} \hatvec{x}
                + \frac{\partial T}{\partial y} \hatvec{y}
                + \frac{\partial T}{\partial z} \hatvec{z}
\end{align*}
where:
\begin{itemize}
  \item $T : \reals^3 \rightarrow \reals$ is a scalar function that is
    continuous and differentiable
\end{itemize}
Some properties of the gradient:
\begin{itemize}
  \item $\del T(x, y, z)$ is the gradient evaluated at a position
    given by $(x, y, z)$.  The vector points in a direction that
    function $T$ increases most quickly.
  \item The function $T$ can be approximated at points $\vect{r}$ near
    $\vect{r}_0 = (x, y, z)$ by the linear function $T(\vect{r}_0) +
    (\vect{r} - \vect{r}_0) \cdot \del T(\vect{r}_0)$
  \item The instantaneous rate of change of $T$ in direction
    $\vect{u}$ (a unit vector) from $\vect{r}_0$ is $\vect{u} \cdot
    \del T(\vect{r}_0)$.  Note that it is always 0 in a direction
    perpendicular to $\del T(\vect{r}_0)$, and the negative of the
    magnitude of $\del T(\vect{r}_0)$ in the opposite direction.
  \item Consider a ``level set'', i.e. a surface defined by all of the
    points $\vect{r}$ where $T(\vect{r})=c$ for some constant $c$.
    Then $\del T$ evaluated at any point on that surface, is normal
    to the surface.
\end{itemize}

If $T$ is a function of other parameters, e.g. of time, then $\del T$
is also a function of those parameters.


\section{Fundamental Theorems of Calculus}

See Section 1.3.2 ``Fundamental Theorem of Calculus'' through Section
1.3.5 ``The Fundamental Theorem for Curls'' of
Griffiths~\cite{Griffiths1998} for some more discussion.

\subsection{The Fundamental Theorem of Calculus}

\begin{equation}
\int_{a}^{b} F(x) \,dx = f(b) - f(a)
\end{equation}
where $F(x) = df/dx$, and $a,b$ are any real numbers.

\subsection{The Fundamental Theorem for Gradients}
\label{sec:ftocgrad}

For any scalar function $T(x,y,z)$ and any two points $\vect{a}$ and
$\vect{b}$:
\begin{equation}
\int_{\vect{a}}^{\vect{b}} (\del T) \cdot d\vect{l} = T(\vect{b}) - T(\vect{a})
\end{equation}

Corollary 1: $\int_{\vect{a}}^{\vect{b}} (\del T) \cdot d\vect{l}$ is
independent of the path taken from $\vect{a}$ to $\vect{b}$.

Corollary 2: $\oint (\del T) \cdot d\vect{l} = 0$, since the beginning
and end points are identical, and hence $T(\vect(b) - T(\vect{a}) =
0$.

\subsection{The Fundamental Theorem for Divergences}
\label{sec:gaussthm}

Also known as Gauss's theorem, Green's theorem, and the divergence
theorem.

For any vector field $\vect{F}$, and any volume $\mathcal{V}$ with
boundary surface $\mathcal{S}$:
\begin{equation}
\int_{\mathcal{V}} (\del \cdot \vect{F}) d\tau = \oint_{\mathcal{S}} \vect{F} \cdot d\vect{a}
\end{equation}

\subsection{The Fundamental Theorem for Curls}
\label{sec:stokesthm}

Also known as Stokes' theorem.

For any vector field $\vect{F}$, and surface $\mathcal{S}$ with
boundary / perimeter path $\mathcal{P}$:
\begin{equation}
\int_{\mathcal{S}} (\del \times \vect{F}) \cdot d\vect{a} = \oint_{\mathcal{P}} \vect{F} \cdot d\vect{l}
\end{equation}

The right-hand side is sometimes called the {\em circulation} of
$\vect{F}$.

Corollary 1: $\int (\del \times \vect{F}) \cdot d\vect{a}$ depends
only on the boundary line, not on the particular surface used.

Corollary 2: $\oint (\del \times \vect{F}) \cdot d\vect{a} = 0$ for
any closed surface, since the boundary line, like the mouth of
balloon, shrinks down to a point, and hence the right hand side of the
equation above vanishes.


\section{Potentials}
\label{sec:potentials}

See Section 1.6.2 ``Potentials'' of Griffiths~\cite{Griffiths1998} for
some more discussion.

A {\em curl-less} or {\em irrotational} vector field $\vect{F}$ is one
where $\del \times \vect{F} = 0$.
Examples include:
\begin{itemize}
\item The electric field in electrostatic scenarios
  (Equation~\eqref{eq:maxfar} when $\vect{B}$ is unchanging over
  time).
\item Gravitational fields in classical mechanics.
\item Constant vector field.
\end{itemize}

\begin{theorem}
\label{thm:irrotationalfieldproperties}
The following conditions are equivalent, i.e. vector field $\vect{F}$
satisfies one if and only if it satsfies all the others:

(a) $\del \times \vect{F} = 0$

(b) $\int_{\vect{a}}^{\vect{b}} \vect{F} \cdot d\vect{l}$ is
independent of path, for any given endpoints $\vect{a}$ and
$\vect{b}$.

(c) $\oint \vect{F} \cdot d\vect{l} = 0$ for any closed loop.

(d) $\vect{F}$ is the gradient of some scalar function:
$\vect{F} = -\del V$.
\end{theorem}

Note that $V$ is not unique.  Any constant can be added to $V$ without
affecting the value of its gradient.
See Appendix~\ref{sec:potentialsthm1} for a proof of
Theorem~\ref{thm:irrotationalfieldproperties}.

A {\em divergence-less} vector field $\vect{F}$ is one where
$\del \cdot \vect{F} = 0$.
Such a field is also called a {\em solenoidal} field,
or an {\em incompressible} field.

The name incompressible comes from fluid mechanics, when studying
fluids that have a constant density.
The origin of the word ``solenoid'' is from Greek root words meaning
``resembling a pipe''.  A wire coiled many times into the shape of the
curved part of a cylinder's surface is shaped like a section of pipe.
The magnetic field created when current flows through a wire in this
shape is divergence-less.  (It turns out that all magnetic fields in
all known scenarios are solenoidal, so this property of magnetic
fields is not restricted to those created by solenoid-shaped current
carrying wires.  Somehow this name became associated with such vector
fields).

Examples of solenoidal fields include:
\begin{itemize}
\item A magnetic field (Equation~\eqref{eq:gaussB}).
\item The velocity field of an incompressible fluid.
\item Current density, in a scenario where the charge density is not
  changing.
\end{itemize}

\begin{theorem}
\label{thm:solenoidalfieldproperties}
The following conditions are equivalent:

(a) $\del \cdot F = 0$

(b) $\int \vect{F} \cdot d\vect{a}$ is independent of surface, for any
given boundary line.

(c) $\oint \vect{F} \cdot d\vect{a} = 0$ for any closed surface.

(d) $\vect{F}$ is the curl of some scalar function:
$\vect{F} = \del \times A$.
\end{theorem}

$A$ is not unique.  The gradient of any scalar function can be added
to $A$ without affecting the curl, since the curl of a gradient is 0.
See Appendix~\ref{sec:potentialsthm2} for
a proof of Theorem~\ref{thm:solenoidalfieldproperties}.

\begin{theorem}[Helmholtz decomposition]
\label{thm:helmholtzdecomposition}
All vector fields $\vect{F}$ can be written as the gradient of a
scalar plus the curl of a vector:
\begin{equation}
\vect{F} = - \del V + \del \times A
\end{equation}
\end{theorem}
TODO: What is the proof of Theorem~\ref{thm:helmholtzdecomposition}?
It might be on this page~\cite{WikipediaHelmholtzDecomposition}, or
some page linked from it.

TODO: Are there ``turn-the-crank'' methods of finding a Helmholtz
decomposition of a vector field, similar to taking one-variable
derivatives?  Or is it more like taking one-variable integrals, where
the result can be much more challenging to find?  Do tools like
Mathematica or open source alternatives have features to help find
them?

TODO: Add ``The Helmholtz Theorem'' here (the one that given a scalar
field that is the divergence of $F$, and a vector field $C$ that is
the curl of $F$, and suitable boundary conditions, we can uniquely
determine $F$.

TODO: Give some examples of both Helmholtz decomposition, and of the
Helmholtz theorem mentioned above.  For Helmholtz decomposition, note
that it is trivial for any irrotational or solenoidal field, because
either the divergence or the curl is 0, so one of the two components
can be 0.  Find more interesting examples than that.

TODO: Same questions as above about whether there are
``turn-the-crank'' methods of finding vector field starting from its
desired divergence and curl.

TODO: Is there any kind of deep relationship to the Helmholtz theorem
and Helmholtz decomposition?

TODO: Give some examples of $\phi$ and $\vect{A}$ representations of
electromagnetic fields in some representative situations, perhaps for
a few different gauges, e.g. Coulomb, Lorenz, Fock-Schwinger.


\section{History of vectors and operations on them}

Before vectors were invented, quaternions were invented by William
Rowan Hamilton.

Some articles and videos related to this topic:

\begin{itemize}

\item James Propp's ``Hamilton's Quaternions, or, The Trouble with
  Triples''~\cite{HamiltonsQuaternions}

\item James Propp's ``Twisty Numbers for a Screwy
  Universe''~\cite{TwistyNumbers} (perhaps not as relevant directly to
  quaternions and vectors, but complex numbers are a fairly direct
  precursor to quaternions).

\item 3Blue1Brown video ``Visualizing quaternions (4d numbers) with
  stereographic projection~\cite{VisualizingQuaternions}.  This is not
  so much about the history of quaternions, as it is understanding
  them.

\end{itemize}


\section{Galilean relativity}

In classical Newtonian mechanics, physical laws are invariant under
Galilean relativity, i.e. any laws that hold in one inertial frame
also hold in all other inertial frames, i.e. those that move with a
constant velocity relative to the first frame.

This applies to Newton's three laws of motion, for example.

It does not apply to measurements of quantities that depend upon
velocity, e.g. an object's momentum or kinetic energy.

However, the properties that momentum is conserved, and energy is
conserved, is preserved across inertial frames, even if the absolute
values of the kinetic energy and momentum differ between frames.

If there are forces that are velocity-dependent, e.g. a frictional
force resisting an object traveling through a medium like air or water
that is a function of velocity relative to that medium, then as long
as that force is always written as a function of the relative velocity
between the object and the medium, then note that the measurements of
such relative velocities are the same in different inertial frames.

For other examples and some mathematical derivations,
see~\cite{GalileanTransformation}.

See this video~\cite{GalileanInvarianceMaxwellsEquations} that gives a
similar mathematical demonstration that Newton's second law is
invariant under Galilean relativity, but Maxwell's equations are {\em
  not}.


\section{Experiments}

\begin{itemize}
\item 1848-49: Fizeau's measurement of the speed of light in
  air~\cite{Fizeau1849}
\item 1851: Fizeau's experiment to measure the relative speed of light
  in a moving medium (water)~\cite{Fizeau1851}
\end{itemize}


\section{Questions}

\begin{itemize}
\item Historically, how was aberration of light
  distinguished/separated from other effects like parallax?
\end{itemize}

Questions that arose for me or others raised in some comments on
Veritasium's video ``Why No One Has Measured The Speed Of
Light''~\cite{OneWaySpeedOfLight}:
\begin{itemize}
\item Uniformity of Cosmic Microwave Background (CMB) radiation is
  evidence in some way that speed of light is close to same in all
  directions?
\item Roemer's measurement of time variation of motions of Jupiter's
  moons.  How is that not some kind of indication of something about
  the one-way speed of light?
\item How does GPS work?  Does it assume one-way speed of light is $c$
  in all directions?
\end{itemize}


\section{Index of A History of the Theories of Aether and Electricity}

All page numbers are of the form ``I(number)'' for a page number in
Volume I, or ``II(number)'' for a page in Volume II.

Done for I, pp. 1-102

\begin{itemize}
\item aberration of light, adumbration by Pardies: I22
\item aberration of light, discovery by Bradley: I94
\item action at a distance: I5, I28, I30-31, I36, I48, I50, I51, I59
\item action, as in principle of least action: I97
\item Aepinus, Franz Ulrich: I51
\item aether: I5, I13, I19, I20, I30, I95, I98
\item Amp\`{e}re, Andr\'{e}-Marie: I83
\item ammunitionm, finite, argument against corpuscular theory of light: I97
\item atomism: I13
\item Bartholin, Erasmus: I25
\item Berzelius, J\"{o}ns Jacob: I78
\item Biot, Jean-Baptiste: I82
\item birefringence, discovered 1669: I25
\item Bologna stone: I20
\item Bradley, James: I94
\item Carlisle, Anthony: I73
\item Cavendish, Henry: I54
\item Cavendish, Charles: I54
\item colors of light, Descartes' theory: I9
\item colors of light, Hooke's theory: I16
\item colors of light, Newton's theory: I17
\item colors of thin plates: I13, I21, I102
\item conductor, origin of name: I42
\item corpuscular theory of light: I10, I20, I23, I31, I97
\item corpuscular theory of light, argument against: I97, I100
\item Coulomb, Augustin: I57
\item Davy, Humphry: I74
\item de Maricourt: see entry alphabetized as ``Maricourt''
\item Desaguliers, Jean Th\'{e}ophile: I41
\item diamond emitting light when rubbed or heated: I14
\item dielectric: I52
\item diffraction: I14, I20, I103
\item du Fay: see entry alphabetized as ``Fay''
\item epistemological rationalism: I7
\item electric charge, conservation of: I47
\item electric charge, induction of: I52
\item electric current effect on magnets: I81-83
\item electric current effect on magnets, and vice versa: I83
\item electric current carrying wires attract or repel: I83
\item electric current through human body, used as sensory evidence: I45, I55, I67, I71, I73
\item electric current through wire, similarity with flow of heat through wire: I91
\item electric fluid: I42
\item electric fluid, one-fluid theory: I50
\item electric fluid, two-fluid theory: I58
\item electric force, how it varies with distance: I53, I54, I57
\item electric fluid in conductor is all at surface at equilibrium: I51, I53, I61
\item electric pile: I71-73
\item electric pile, experiments varying the strength of: I89
\item electrolysis of water and other chemicals, both by continuous current and frictional electricity: I74
\item electrical conducting power of different materials and shapes, generated by frictional electricity: I55
\item electrical conducting power of different materials, lengths, cross-sectional areas, and temperatures, generated by electric piles: I90
\item electrification, discovery of two kinds by du Fay in 1733: I44
\item electrification only at the surface, discovery in 1729: I43, I49
\item electrical emanations: I36
\item electricity, discoveries by Gilbert, and naming: I35
\item electricity, discovery of transfer between electrified bodies: I42
\item electricity, ordinary matter (without electric fluid) repels other ordinary matter: I52
\item electricity, resinous: I44
\item electricity, theory of electric fluid that electrified bodies have a surplus or deficiency of when electrified: I46, I49
\item electricity, vitreous: I44
\item electrodynamics, origin of name: I85
\item electrodynamics, Amp\`{e}re's theories and mathematical analysis of: I83-88
\item electrostatic mathematics by Poisson: I61-62
\item extraordinary ray: see birefringence
\item du Fay, Charles-Fran\c{c}ois: I43
\item fits of easy transmission and easy reflection, theory of, about light: I22
\item Franklin, Benjamin: I45, I47
\item frogs in electrical experiments: I67-69
\item Galvani, Luigi: I67
\item galvanism, i.e. electric current caused by touching of different metals: I70
\item galvanism, proof tha it was electrical in nature: I71
\item Gilbert, William: I34
\item Gravesande: see entry alphabetized as ``'s Gravesande''
\item Gray, Stephen: I41
\item Green, George: I65
\item Grene's theorem: I65
\item gravity: I28, I31
\item heat: I20, I38
\item heat, from focused light: I23
\item heat vs. light, heat absorbed by glass but not light: I40
\item heat vs. light, heat transmitted by different colors of light: I99
\item Herschel, Sir William: I99
\item Huygens' principle of light wave propagation: I24
\item induction of electric charge: see electric charge, induction of
\item infrared light: I99
\item interference of waves, applied to water and light: I101
\item interference of light waves, applied to colors of thin plates: I102
\item inter-phenomena: I30
\item Leyden phial, Leyden jar: I45
\item magnet, use as a compass, as early as 1186 in Europe: I33
\item magnet, Earth is a magnet: I34
\item magnetic fluid, two-fluid theory: I57, I59
\item magnetic force between poles decreases as square of distance: I56, I59
\item magnetic moment: I63
\item magnetic poles, discovery and naming: I33
\item magnetism, discoveries by Gilbert: I35
\item magnetism, distinguishing behavior from gravity: I55
\item magnetism, created by lightning: I80-81
\item de Maricourt, Pierre, also known as Peregrinus: I33
\item de Maupertuit, Peirre-Louis-Moreau: I96
\item mechanical philosophy: I6
\item Michell, John: I56
\item van Musschenbroek, Pieter: I45
\item Neckam, Alexander: I33
\item Newton's rings: I13, I102
\item Nicholson, William: I73
\item Oersted, Hans Christian: I80
\item Ohm, Georg Simon: I90
\item Ohm's law: I92
\item opacity: I25
\item plenum: I5, I13
\item Poisson, Sim\'{e}on Denis: I60
\item polarized light: I19, I27
\item poles of magnets: see magnetic poles
\item potential: I65
\item Priestley, Joseph: I53
\item principle of least time: I12
\item reflection of light: I10, I24
\item refraction of light: I10, I15, I17, I20, I25, I95, I101
\item resinous: see electricity, resinous
\item Ritter, Johann Wilhelm: I74, I76
\item Roemer, Olaf: I22
\item 's Gravesande, Wilhelm Jacob: I37
\item Savart, F\'{e}lix: I82
\item Snell, Willebrord: I10
\item sound as periodic vibrational waves: I21
\item speed of light propagation: I22
\item thermo-electric circuit: I89
\item Torricelli: I24
\item ultraviolet light: I100
\item wave theory of light: I14, I23
\item Wilcke, Johan Carl: I51
\item vacuum, light traveling through: I24
\item van Musschenbroek: see entry alphabetized as ``Musschenbroek''
\item vitreous: see electricity, vitreous
\item void: I5, I13
\item Volta, Alessandro: I69
\item voltaic polarization: I76
\item Watson, William: I45
\item wave theory of light: I100-101
\item waves, interference of: see ``interference''
\item Young, Thomas: I100
\end{itemize}

\subsection{Definitions of some key terms}

\paragraph{aberration of light}

\paragraph{diffraction of light}

Diffraction of light is the bending of light waves as they pass around
the edges of an obstacle or through a narrow opening.  This bending
causes the light to spread out and illuminate areas where a shadow
would otherwise be expected.  Diffraction is a wave phenomenon,
meaning it's a characteristic behavior of waves, including light,
sound, and water waves.

Bending Around Obstacles: When light encounters an obstacle, like the
edge of a barrier or a small opening, it doesn't just travel in a
straight line.  Instead, the light waves spread out and bend around
the obstacle, a phenomenon known as diffraction.

Interference: Diffraction often occurs alongside interference, where
waves combine, either constructively (creating brighter areas) or
destructively (creating darker areas).  This combination of diffraction
and interference is responsible for the characteristic patterns
observed in diffraction experiments.

Wavelength Dependence: The extent of diffraction depends on the
wavelength of the light and the size of the opening or obstacle.  When
the wavelength of the light is comparable to the size of the opening
or obstacle, diffraction is more pronounced.

Examples: A classic example is the single-slit experiment where light
passes through a narrow opening and creates a diffraction pattern on a
screen behind it.  Another example is the silver lining seen around
clouds in the sky, caused by sunlight diffracting through water
droplets or ice crystals.


\paragraph{diffusion of light}

Light diffusion is the process where light scatters or spreads out in
various directions, rather than traveling in a straight line.  This
happens when light encounters a medium or surface with irregularities,
causing it to scatter at different angles.  This scattering results in
a softer, more even light, reducing harsh shadows and glare.

Scattering: When light hits a rough or uneven surface (like a matte
surface or a translucent material), it scatters in multiple directions
instead of reflecting at a single angle.

Softening: This scattering effect diffuses the light, making it appear
softer and more gentle.

Reduced glare: By scattering the light, diffusion reduces the
intensity of the light source, minimizing glare and harsh shadows.

Examples:
\begin{itemize}
\item Transmission diffusion: Light passing through a frosted glass or
  plastic panel.
\item Reflection diffusion: Light bouncing off a whiteboard or
  textured wall.
\item Softboxes and diffusers: Professional tools like softboxes and
  diffusion panels are used in photography and filmmaking to soften
  and spread out light.
\item Overcast days: The clouds act as a natural diffuser, scattering
  sunlight and creating soft, even light.
\end{itemize}

\paragraph{dispersion of light}

Dispersion of light is the phenomenon where white light, upon passing
through a medium like a prism, separates into its constituent colors
(like in a rainbow).  This happens because different wavelengths
(colors) of light bend at slightly different angles when they enter
and exit the medium.  When light passes from one medium to another
(like from air to glass), it bends (refracts).  The amount of bending
depends on the wavelength of the light.  Shorter wavelengths (like
violet and blue) bend more than longer wavelengths (like red and
orange).

\paragraph{induction of electric charge}

\paragraph{induction of electric current}

\paragraph{interference}

\paragraph{polarization of light}

\paragraph{reflection}

\paragraph{refraction}

Refraction of light is the bending of light as it passes from one
medium to another, caused by a change in the speed of light.  This
bending occurs because light travels at different speeds in different
materials. For example, light travels slower in water than in air,
causing it to bend when entering or exiting water.


\subsection{Noteworthy events in aether vs. action at a distance}

Vol. I, p. 28:

\begin{quote}
  Newton claimed nothing more for his discovery than that it provided
  the necessary instrument for methematical prediction, and he pointed out
  that it did not touch on the question of the mechanism of gravity.
  As to this, he conjectured that the density of the aether might vary from
  place to place, and that bodies might tend to move from the denser parts
  of the medium toward the rarer; but whether this were the true explanation
  or not, at any rate, he said, to suppose 'that one body may act upon
  another at a distance through a vacuum, without the mediation of anything
  else, ... is to me so great an absurdity, that I believe no man, who has
  in philosophical matters a competent faculty for thinking, can ever fall
  into.'
\end{quote}

According to Wikipedia~\cite{WikipediaNewtonLawOfGravity}, Newton
wrote this in 1692 in a letter to Richard Bentley, 5 years after
Principia was first published in 1687.  I do not know if Newton said
anything similar in Principia itself.  Very likely he did not, or else
it would be more widely quoted on this topic.

Vol. I, p. 30:

\begin{quote}
  The rejection of the inverse-square law of gravitation by the French
  Cartesians antagonised the younger disciples of Newton to such an
  extent that the latter hardened into opposition not only to the
  vortices but to the whole body of Cartesian notions, including the
  aether.  In the second edition (1713) of the {\em Principia}, there
  is a preface written by Roger Cotes (1682-1716), in which the
  Newtonian law of action at a distance is championed as being the
  only formulation of the facts of experience which does not introduce
  unverifiable and useless suppositions.
\end{quote}

Vol. I, p. 31:

\begin{quote}
  When the eighteenth-century natural philosophers found by experience
  that the Newtonian law was marvellously powerful, yielding formulae
  by which practically every observable motion in the solar system
  could be predicted, while on the other hand the search for an
  explanation of inter-phenomena led to no practical results, opinion
  set in favour of Cotes's attitude, which came to prevail widely; and
  in the middle of the century R. G. Boscovich (1711-87) ... attempted
  to account for all known physical effects in terms of action at a
  distance between point particles.
\end{quote}

%\newpage
\bibliography{refs}

\appendix

\section{Proof that curl of a gradient is 0}
\label{sec:curlofgradis0}

\subsection{Proof from Griffith's definition of curl and gradient}
\label{sec:curofgradis0proof1}

Definition of a gradient for any scalar field $T$:
\begin{align*}
  \del T & \equiv \frac{\partial T}{\partial x} \hatvec{x}
                + \frac{\partial T}{\partial y} \hatvec{y}
                + \frac{\partial T}{\partial z} \hatvec{z}
\end{align*}
Definition of a curl for any vector field $\vect{v}$:
\begin{align*}
  \del \times \vect{v} &
      \equiv \begin{vmatrix}
                 \hatvec{x} & \hatvec{y} & \hatvec{z} \\
                 \partial / \partial x & \partial / \partial y & \partial / \partial z \\
                 v_x & v_y & v_z
             \end{vmatrix} \\
      & =   \hatvec{x} ( \frac{\partial v_z}{\partial y} - \frac{\partial v_y}{\partial z} )
          + \hatvec{y} ( \frac{\partial v_x}{\partial z} - \frac{\partial v_z}{\partial x} )
          + \hatvec{z} ( \frac{\partial v_y}{\partial x} - \frac{\partial v_x}{\partial y} )
\end{align*}
First, simply apply the definitions to $\del \times (\del T)$.  Then
do a little bit of algebra:
\begin{align}
\del \times (\del T) &
  = \hatvec{x} ( \frac{\partial}{\partial y} ( \frac{\partial T}{\partial z} ) - \frac{\partial}{\partial z} ( \frac{\partial T}{\partial y} ) )
  + \hatvec{y} ( \frac{\partial}{\partial z} ( \frac{\partial T}{\partial x} ) - \frac{\partial}{\partial x} ( \frac{\partial T}{\partial z} ) )
  + \hatvec{z} ( \frac{\partial}{\partial x} ( \frac{\partial T}{\partial y} ) - \frac{\partial}{\partial y} ( \frac{\partial T}{\partial x} ) )
  \label{eqn:curlofgrad1}
\end{align}

As long as $T$ satisfies appropriate conditions, then Clairaut's
theorem (aka Schwartz's theorem) says that regardless of which order
you take partial derivatives, the final result is equal.  I believe a
stronger condition than necessary is that $T$ has continuous second
partial derivatives everywhere.
See~\cite{SymmetryOfSecondDerivatives} for more precise conditions.

As long as taking second partial derivatives of $T$ by variables in
different orders gives the same result, we can see that the right hand
side of~\eqref{eqn:curlofgrad1} must be 0.
\begin{equation}
\del \times (\del T) = 0
\end{equation}


\subsection{A proof based on Stokes' Theorem}

This alternate is intended to perhaps give a little more insight into
{\em why} the curl of the gradient is always 0.

Let $T$ be any scalar field, and $\vect{F} = \del T$ be its gradient.
By the fundamental theorem for gradients (Section~\ref{sec:ftocgrad})
Corollary 2, $\oint \vect{F} \cdot d\vect{l} = 0$ for all closed
paths.

By Stokes' theorem (Section~\ref{sec:stokesthm}):
\begin{align*}
\int_{\mathcal{S}} (\del \times \vect{F}) \cdot d\vect{a} & = \oint_{\mathcal{P}} \vect{F} \cdot d\vect{l} \\
   & = 0
\end{align*}

By Lemma~\ref{lem:techlemma1} (see Section~\ref{sec:techlemma1}),
The only way that the flux of $\del \times \vect{F}$ can be 0 for all
surfaces $\mathcal{S}$ is if $\del \times \vect{F} = 0$ everywhere.


\subsection{Technical lemma}
\label{sec:techlemma1}

\begin{lemma}
\label{lem:techlemma1}
Let $\vect{F}$ be a continuous vector field where
\begin{equation}
\int_{\mathcal{S}} \vect{F} \cdot d\vect{a} = 0
\end{equation}
for all surfaces $\mathcal{S}$.  Then $\vect{F} = 0$.
\end{lemma}

\begin{proof}
Performing a proof by contradiction, we suppose that there is a point
$\vect{p}$ where $\vect{F}(p) \neq 0$, and deduce that there must be a
surface where $\int_{\mathcal{S}} \vect{F} \cdot d\vect{a} \neq 0$.

Let $\vect{u}$ be the unit vector
$\vect{F}(\vect{p}) / |\vect{F}(\vect{p})|$,
and define the scalar field
$G(\vect{x}) = \vect{F}(\vect{x}) \cdot \vect{u}$.
$G$ is continuous, and $G(\vect{p}) = |\vect{F}(\vect{p})| = c > 0$.

By continuity of $G$, there is an open neighborhood $V$ of $\vect{p}$
such that $G(\vect{x}) \geq c/2$ for all points $\vect{x}$ in that neighborhood.
Let surface $\mathcal{S}$ be a circular disc contained entirely
inside of $V$, centered at $\vect{p}$, with normal $\vect{u}$.
Then:
\begin{align*}
\int_{\mathcal{S}} \vect{F} \cdot d\vect{a}
     = & \int_{\mathcal{S}} G(\vect{x}) ds \\
  \geq & \int_{\mathcal{S}} (c/2) ds \\
     = & (c/2) Area(\mathcal{S}) \\
     > & 0
\end{align*}
\end{proof}


\section{Proof that divergence of a curl is 0}
\label{sec:divofcurlis0}

\subsection{Proof from Griffith's definition of divergence and curl}

There is a straightforward proof of $\del \cdot (\del \times \vect{F})
= 0$ for all vector fields that follows from the definitions, with
only a little bit of algebra and a use of Clairaut's/Schwartz's
theorem, very similar to the proof given in
Appendix~\ref{sec:curofgradis0proof1}.  I won't write it out
explicitly here unless it becomes very useful to have as a reference,
because it is very straightforward, and does not provide much insight
into why it is true.


\subsection{A proof based on Stokes' and Gauss's Theorems}

Let $\vect{F}$ be any vector field, and $\vect{G} = \del \times
\vect{F}$ its curl.
By Stokes' theorem (Section~\ref{sec:stokesthm}) Corollary 2:
\begin{equation}
\oint_{\mathcal{S}} \vect{G} \cdot d\vect{a} = 0
\end{equation}
for any closed surface $\mathcal{S}$.
By Gauss's theorem (Section~\ref{sec:gaussthm}):
\begin{align*}
\int_{\mathcal{V}} (\del \cdot \vect{G}) d\tau
    & = \oint_{\mathcal{S}} \vect{G} \cdot d\vect{a} \\
    & = 0
\end{align*}
It is a fundamental result in calculus that if the integral of a
continuous function over every volume is zero, then the function
itself must be identically zero.  This is often referred to as a
consequence of the Fundamental Lemma of the Calculus of
Variations~\cite{FundamentalLemmaCalculusOfVariations}.
Thus $\del \cdot \vect{G} = \del \cdot (\del \times F)$ is identically
0.


\section{Proof of theorem about irrotational fields}
\label{sec:potentialsthm1}

That is, a proof of Theorem~\ref{thm:irrotationalfieldproperties} from
Section~\ref{sec:potentials}.
This proof is essentially the same as one given in Section 2.3.1
``Introduction to Potential'' of~\cite{Griffiths1998}.

Let $\vect{F}$ be a vector field such that $\del \times \vect{F} = 0$.
Stokes' theorem says, for all vector fields $\vect{F}$:
\begin{equation}
\int_{\mathcal{S}} (\del \times \vect{F}) \cdot d\vect{a} = \oint_{\mathcal{P}} \vect{F} \cdot d\vect{l} \nonumber
\end{equation}
Because $\del \times \vect{F} = 0$, the left-hand side is 0 for all
surfaces $\mathcal{S}$.  Thus for all paths $\mathcal{P}$ that are
boundaries of appropriate surfaces $\mathcal{S}$, the right-hand side
is also 0.
\begin{equation}
\oint_{\mathcal{P}} \vect{F} \cdot d\vect{l} = 0 \label{eq:thm1allpathintegrals0}
\end{equation}

I will call such paths ``Stokes paths'' here, for brevity (TODO: It
might be nice to give the full set of restrictions that a Stokes path
must satisfy).
This proves that (a) implies (c) for
Theorem~\ref{thm:irrotationalfieldproperties}, at least for closed
loops that are Stokes paths.

Consider an arbitrary distinct pair of points $\vect{a}$ and
$\vect{b}$, and any closed loop Stokes path $\mathcal{P}$ that
includes both of those points.  Let $\mathcal{P}_1$ be the part of
$\mathcal{P}$ from $\vect{a}$ to $\vect{b}$, and $\mathcal{P}_2$ be
the remaining part of the path from $\vect{b}$ back to $\vect{a}$.
Let $\bar{\mathcal{P}_2}$ be the same as $\mathcal{P}_2$, but in the
reverse direction.  Then:
\begin{align}
  0
      & = \oint_{\mathcal{P}} \vect{F} \cdot d\vect{l} & & \text{by~\eqref{eq:thm1allpathintegrals0}} \nonumber \\
      & = \int_{\mathcal{P}_1} \vect{F} \cdot d\vect{l} + \int_{\mathcal{P}_2} \vect{F} \cdot d\vect{l} & & \text{break $\mathcal{P}$ into two subpaths} \nonumber \\
      & = \int_{\mathcal{P}_1} \vect{F} \cdot d\vect{l} - \int_{\bar{\mathcal{P}_2}} \vect{F} \cdot d\vect{l} & & \text{line integral of reverse path is negative of forward path} \nonumber
\end{align}
Therefore:
\begin{equation}
\int_{\mathcal{P}_1} \vect{F} \cdot d\vect{l} = \int_{\bar{\mathcal{P}_2}} \vect{F} \cdot d\vect{l}
\end{equation}
Thus for any pair of Stokes paths $\mathcal{P}_1$ and
$\bar{\mathcal{P}_2}$ from $\vect{a}$ to $\vect{b}$, their path
integrals are equal.
This proves that (c) implies (b) for
Theorem~\ref{thm:irrotationalfieldproperties}, for Stokes paths.

Now consider (b) as our starting point.  Pick some arbitrary point
$\vect{d}$.  Define the scalar function $V(\vect{r})$ as:
\begin{equation}
V(\vect{r}) = - \int_{\vect{d}}^{\vect{r}} \vect{F} \cdot d\vect{l}
\end{equation}
Because of (b), the particular path over which these line integrals
are performed does not affect the result.
\begin{align*}
V(\vect{b}) - V(\vect{a})
  = & - \int_{\vect{d}}^{\vect{b}} \vect{F} \cdot d\vect{l}
      + \int_{\vect{d}}^{\vect{a}} \vect{F} \cdot d\vect{l} \\
  = & - \int_{\vect{a}}^{\vect{d}} \vect{F} \cdot d\vect{l}
      - \int_{\vect{d}}^{\vect{b}} \vect{F} \cdot d\vect{l} \\
  = & - \int_{\vect{a}}^{\vect{b}} \vect{F} \cdot d\vect{l}
\end{align*}
By the fundamental theorem for gradients (Section~\ref{sec:ftocgrad}):
\begin{equation}
\int_{\vect{a}}^{\vect{b}} (\del V) \cdot d\vect{l} = V(\vect{b}) - V(\vect{a})
\end{equation}
Thus:
\begin{equation}
       \int_{\vect{a}}^{\vect{b}} (\del V) \cdot d\vect{l}
   = - \int_{\vect{a}}^{\vect{b}} \vect{F} \cdot d\vect{l}
\end{equation}
Since this is true for all pairs of points $\vect{a}$ and $\vect{b}$,
the functions being integrated must be equal:
\begin{equation}
\vect{F} = - \del V
\end{equation}
This proves that (b) implies (d) for
Theorem~\ref{thm:irrotationalfieldproperties}, for Stokes paths.

Section~\ref{sec:curlofgradis0} contains the proof that (d) implies
(a).


\section{Proof of theorem about solenoidal fields}
\label{sec:potentialsthm2}

That is, a proof of Theorem~\ref{thm:solenoidalfieldproperties} from
Section~\ref{sec:potentials}.
This proof is strongly hinted at by Problem 5.31 in Section 5.4.1
``The Vector Potential'' of~\cite{Griffiths1998}.

We are given a vector field $\vect{F}$ such that
$\del \cdot \vect{F} = 0$.
By Gauss's theorem (Section~\ref{sec:gaussthm}),
for any volume $\mathcal{V}$ with boundary surface $\mathcal{S}$:
\begin{equation}
\int_{\mathcal{V}} (\del \cdot \vect{F}) d\tau = \oint_{\mathcal{S}} \vect{F} \cdot d\vect{a}
\end{equation}
Since $\del \cdot \vect{F} = 0$, the left hand side is always 0.
So:
\begin{equation}
\oint_{\mathcal{S}} \vect{F} \cdot d\vect{a} = 0
\end{equation}
This proves that (a) implies (c) for
Theorem~\ref{thm:solenoidalfieldproperties}.


TODO: Proof that (c) implies (b).  Basic idea: Given a boundary line,
pick an arbitrary closed surface that completely contains that
boundary line.  By (b) we know that the total flux through the entire
closed surface is 0.  Assuming the boundary line divides that closed
surface into 2 non-closed surfaces (TODO: I assume that some
conditions must hold for the boundary line in order for that to be
true, at least that the boundary line is simple?) $S_1$ and $S_2$, the
flux through $S_1$ must be the negative of the flux through $S_2$.  We
can replace $S_2$ with any other surface $S_3$ such that $S_1 \cup
S_3$ is a closed surface, and the flux through that new closed surface
is also 0, so the flux through $S_3$ must be the same as that through
$S_2$, and $S_3$ has the same boundary line as $S_2$.

TODO: Proof that (b) implies (d) is (I hope) what Griffiths Problem
5.31 helps me to flesh out in more detail.

Section~\ref{sec:divofcurlis0} contains the proof that (d) implies
(a).


\end{document}
