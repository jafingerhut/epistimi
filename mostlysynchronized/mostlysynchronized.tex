\documentclass[a4paper]{article}

\usepackage[pages=all, color=black, position={current page.south}, placement=bottom, scale=1, opacity=1, vshift=5mm]{background}
%\SetBgContents{
%	\tt This work is shared under a \href{https://creativecommons.org/licenses/by-sa/4.0/}{CC BY-SA 4.0 license} unless otherwise noted
%}      % copyright

\usepackage[margin=1in]{geometry} % full-width

% AMS Packages
\usepackage{amsmath}
\usepackage{amsthm}
\usepackage{amssymb}

% Unicode
\usepackage[utf8]{inputenc}
\usepackage{hyperref}
\hypersetup{
	unicode,
%	colorlinks,
%	breaklinks,
%	urlcolor=cyan,
%	linkcolor=blue,
	pdfauthor={Author One, Author Two, Author Three},
	pdftitle={A simple article template},
	pdfsubject={A simple article template},
	pdfkeywords={article, template, simple},
	pdfproducer={LaTeX},
	pdfcreator={pdflatex}
}

% Natbib
\usepackage[sort&compress,numbers,square]{natbib}
\bibliographystyle{mplainnat}

% Theorem, Lemma, etc
\theoremstyle{plain}
\newtheorem{theorem}{Theorem}
\newtheorem{corollary}[theorem]{Corollary}
\newtheorem{lemma}[theorem]{Lemma}
\newtheorem{claim}{Claim}[theorem]
\newtheorem{axiom}[theorem]{Axiom}
\newtheorem{conjecture}[theorem]{Conjecture}
\newtheorem{fact}[theorem]{Fact}
\newtheorem{hypothesis}[theorem]{Hypothesis}
\newtheorem{assumption}[theorem]{Assumption}
\newtheorem{proposition}[theorem]{Proposition}
\newtheorem{criterion}[theorem]{Criterion}
\theoremstyle{definition}
\newtheorem{definition}[theorem]{Definition}
\newtheorem{example}[theorem]{Example}
\newtheorem{remark}[theorem]{Remark}
\newtheorem{problem}[theorem]{Problem}
\newtheorem{principle}[theorem]{Principle}

\usepackage{graphicx, color}
\graphicspath{{fig/}}
\setlength{\abovecaptionskip}{-5pt}

%\usepackage[linesnumbered,ruled,vlined,commentsnumbered]{algorithm2e} % use algorithm2e for typesetting algorithms
\usepackage{algorithm, algpseudocode} % use algorithm and algorithmicx for typesetting algorithms
\usepackage{mathrsfs} % for \mathscr command

%%%%%%%%%%%%%%%%%%%%%%%%%%%%%%%%%%%%%%%%%%%%%%%%%%%%%%%%%%%%%%%%%%%%%%
% Experiment to create a table with arrows between selected pairs of
% cells, copied from:
% https://tex.stackexchange.com/questions/163174/table-with-arrows-between-cells
\usepackage{tabularx}
\usepackage{tikz}
%\tikzset{every picture/.style={remember picture}}

\renewcommand\tabularxcolumn[1]{m{#1}}
\newcolumntype{M}{>{\centering\arraybackslash}m{1cm}}
\newcolumntype{Y}{>{\centering\arraybackslash}X}

\newcommand\tikzmark[2]{%
\tikz[remember picture,baseline] \node[inner sep=2pt,outer sep=0] (#1){#2};%
}

\newcommand\link[2]{%
\begin{tikzpicture}[remember picture, overlay, >=stealth, shift={(0,0)}]
  \draw[->] (#1) to (#2);
\end{tikzpicture}%
}
%%%%%%%%%%%%%%%%%%%%%%%%%%%%%%%%%%%%%%%%%%%%%%%%%%%%%%%%%%%%%%%%%%%%%%

% Author info
\title{
  Mostly Synchronized \\
  \large How to synchronize clocks almost, but not quite (... entirely
  unlike tea)
  \footnote{With a hat tip to Billy Crystal in ``The Princess Bride'' and ``The Hitchhiker's Guide to the Galaxy'' by Douglas Adams}}
\author{J. Andrew Fingerhut (\texttt{andy.fingerhut@gmail.com})}

\date{
        August 6, 2025
%	\today
}

%\newcommand{\ihat}{\textbf{i}}
%\newcommand{\jhat}{\textbf{j}}
%\newcommand{\khat}{\textbf{k}}
%\newcommand{\rhat}{\hat{\textbf{r}}}
%\newcommand{\vect}[1]{\textbf{#1}}
%\newcommand{\hatvec}[1]{\hat{\textbf{#1}}}

\newcommand{\ihat}{\mathbf{i}}
\newcommand{\jhat}{\mathbf{j}}
\newcommand{\khat}{\mathbf{k}}
\newcommand{\rhat}{\hat{\mathbf{r}}}
\newcommand{\vect}[1]{\mathbf{#1}}
\newcommand{\hatvec}[1]{\hat{\mathbf{#1}}}

\begin{document}
\maketitle

%\begin{abstract}
%  abstract here
%\end{abstract}

\tableofcontents

\section{Introduction}
\label{sec:intro}

There have been many attempts made to devise methods by which two
clocks, separated by some noticeable distance, can be synchronized so
closely to one another, that you can use them to make a measurement of
the speed of light in one direction.  A recent video describing some
of these attempts was published by Veritasium in
2021~\cite{OneWaySpeedOfLight}.

There are many methods to synchronize clocks that are {\em not}
precise enough to measure the one way speed of light, e.g. methods
used in computer networks like the Network Time Protocol
(NTP)~\cite{NetworkTimeProtocol} and {\sc Huygens}~\cite{GLYN2018}.
These involve measuring the time on a local clock, sending it in
messages to other computer systems, and sometimes performing extensive
calculations on the contents of sequences of these messages, in order
to get closer to a level of synchronization precision that is possible
for causal messages passed via signals traveling at the speed of
light, or usually slower.

Those methods will not be discussed further in this document.


\section{Methods that seem like they might be able to allow light-precise clock sychronization}

\subsection{Synchronize stationary clocks with known distance between them by constant-speed traveler}

I heard about this from James Ellias, who may have heard about it from
someone else (Aviram Rosochotsky?).  It would be good to learn who
created it, if it is known, and whether there are any published
references to it.


\subsubsection{First ChatGPT prompt and response}

\paragraph{I asked ChatGPT to comment on it using this prompt:}

Consider the following attempt to synchronize clocks precisely enough
to measure the one way speed of light, and comment on what about it
prevents it from achieving that level of synchronization precision.
Two observers A and B with clocks are connected by a rod of known
length $L$.  A traveler C with a precise clock moves with precisely
known constant speed $v$ parallel to the length of the rod, very close
to the line of the rod.  As C passes near A, they synchronize their
clocks with each other by exchanging light signals.  Later as C passes
near B, they synchronize their clocks.  Both A and B know C's speed
precisely, and so can calculate the Lorentz factor for C's speed $v$,
and thus B can calculate the precise amount to update its clock so
that it becomes the same as A's clock.

\paragraph{ChaptGPT's reponse:}

You're describing a clever attempt to use a moving traveler (C) to
synchronize two spatially separated clocks (A and B) without directly
assuming the one-way speed of light — a core challenge in the
foundations of relativity.  The method you outline cannot succeed at
measuring the one-way speed of light independently, and the reason
lies in an unavoidable circularity tied to the relativity of
simultaneity and Einstein synchronization.

Let’s break this down step by step:

Setup Summary:

\begin{itemize}
\item A and B are at rest in some inertial frame and separated by a
  known distance $L$.
\item C moves at known speed $v$ from A toward B, very close to the
  line connecting them.
\item C passes A, synchronizes clocks with A (e.g., they agree that
  $t_C = t_A = 0$).
\item C continues at constant speed, passes B, and B synchronizes with
  C’s clock.
\item B then computes how to adjust its clock so it is synchronized
  with A’s — using:
  \begin{itemize}
    \item C’s speed $v$
    \item The Lorentz time dilation factor $\gamma$
  \end{itemize}
\end{itemize}

The Goal

To use this process to establish whether clocks A and B are
synchronized in such a way that one can measure the one-way speed of
light (i.e., the time it takes light to go from A to B or B to A
without assuming it's the same in both directions).

Why This Doesn't Work: Circular Dependence on Clock Synchronization

Here’s the core problem:

To determine how much B should adjust its clock using C’s reading and
time dilation, you must already assume a convention for simultaneity
-- which depends on the one-way speed of light.

More precisely:

\begin{enumerate}

\item When C synchronizes with A, that moment is defined in A’s rest
  frame as the clocks being equal -- fine.

\item When C reaches B, its clock shows some time $t_C$, and B wants
  to infer what time A’s clock shows now in A’s rest frame. But here’s
  the catch:

  \begin{itemize}
  \item To determine what time it is at A simultaneously with C’s
    arrival at B, you need a definition of simultaneity.
  \item But simultaneity in special relativity is frame-dependent, and
    to determine simultaneity across space, you need to assume
    something about how signals propagate -- in particular, the
    one-way speed of light.
  \end{itemize}

\item When B applies a correction based on time dilation (i.e.,
  assumes that C's clock ran slow by a factor $\gamma$), that
  correction is valid only under the assumption that A and B are
  already Einstein-synchronized -- i.e., that the one-way speed of
  light is the same in both directions.
\end{enumerate}

The Circular Logic

\begin{itemize}
\item You use time dilation to calculate the time at A from the clock
  reading at C when C reaches B.
\item But time dilation itself -- and the transformation between C's
  time and A/B's time -- depends on the synchronization of clocks A
  and B using Einstein's method.
\item And Einstein synchronization assumes the one-way speed of light
  is $c$.
\end{itemize}

Hence, you're assuming the very thing you're trying to measure.

The Philosophical Core

This goes back to a foundational insight of special relativity (and
philosophy of physics):

The one-way speed of light cannot be measured without already
synchronized clocks -- and clock synchronization requires a convention
(e.g., Einstein synchronization), which assumes the one-way speed of
light is equal in all directions.

So any procedure that seems to measure the one-way speed of light
actually smuggles in that assumption in its setup, typically through
how simultaneity or clock adjustments are defined.


What You Can Measure: The Two-Way Speed of Light

All actual physical measurements of light’s speed measure round-trip
(two-way) times, which are invariant and do not depend on
synchronization conventions.  That’s what experiments like
Michelson-Morley and modern laser-ranging experiments measure -- and
all confirm that the round-trip speed is exactly $c$.

Related Thought Experiments

\begin{itemize}
\item Einstein’s train and platform: Demonstrates relativity of
  simultaneity.
\item Reichenbach’s $\epsilon$-convention: Formalizes the idea that
  you can set the one-way speed of light arbitrarily (as long as
  round-trip speed is $c$).
\end{itemize}

Conclusion

The method you described assumes the one-way speed of light in how B
corrects its clock using C’s time.  That assumption is buried in the
application of Lorentz transformations and time dilation, which
themselves depend on Einstein synchronization.

Therefore, the setup cannot be used to independently measure the
one-way speed of light or synchronize clocks to the required precision
without circularity.

Let me know if you'd like this expressed with spacetime diagrams or
coordinate transformations to make it clearer.

There are some follow up questions and answers in
Appendix~\cite{app:chatgpt-on-clock-sync-and-one-way-speed-of-light}
that I found interesting, but are relevant for other methods besides
this one, hence putting them in an appendix.



%\newpage
\bibliography{refs}

\appendix

\section{Q\&A with ChatGPT I found interesting}
\label{app:chatgpt-on-clock-sync-and-one-way-speed-of-light}


\subsection{Does Einstein's clock synchronization postulate make it impossible to measure one way speed of light?}

\paragraph{I followed up with this prompt:}

Does Einstein's clock synchronization convention postulate effectively
make it impossible to measure the one way speed of light by the very
fact of accepting that postulate?

\paragraph{ChaptGPT's reponse:}

Yes, exactly -- and you've put your finger on a deep and often
misunderstood point.

Short Answer

Yes:

Einstein's clock synchronization convention defines the one-way speed
of light to be equal to the two-way (round-trip) speed of light.  As a
result, once you accept that convention, the one-way speed of light is
not something you can measure independently -- it's a matter of
definition, not discovery.

Why This Is So

Let's step back.

In his 1905 paper, Einstein proposed a convention for synchronizing
clocks at different locations in an inertial frame:

To synchronize two clocks A and B, send a light signal from A to B and
back to A. If the signal takes time $t_1$ to go from A to B and $t_2$
to return, then assume the time for each leg is equal:

\begin{equation}
t_{\text{AB}} = t_{\text{BA}} = \frac{1}{2}(t_1 + t_2)
\end{equation}

This is a postulate, not a derived fact.  It assumes:
\begin{itemize}
\item The one-way speed of light is the same in both directions (A to
  B and B to A).
\item This speed is equal to $c$, the measured two-way speed.
\end{itemize}

So when we say we've synchronized two clocks ``Einstein-style,'' we've
built in the assumption that the one-way speed of light is $c$.

The Circularity

If you now ask:

``Can I test whether the one-way speed of light really is $c$?''

You're faced with this issue:
\begin{itemize}
\item To measure how long it takes for light to go from A to B, you
  need synchronized clocks at A and B.
\item But you synchronized those clocks by assuming the very thing
  you're trying to test -- that light takes the same amount of time in
  both directions.
\item Therefore, any such measurement cannot independently confirm or
  refute the one-way speed of light being $c$; it's conventional.
\end{itemize}

The Conventionality Thesis

This insight has been formalized in philosophy of physics, especially
by thinkers like Hans Reichenbach. He proposed a more general
synchronization scheme using a parameter $\epsilon \in [0,1]$, where:
\begin{equation}
t_{AB} = \epsilon(t_1 + t_2), t_{BA} = (1-\epsilon)(t_1 + t_2)
\end{equation}
\begin{itemize}
\item Einstein synchronization sets $\epsilon = 0.5$
\item But any $\epsilon \in [0,1]$ leads to a logically consistent
  theory, as long as the two-way speed is still $c$.
\end{itemize}
So the one-way speed of light is not an observable quantity -- it's a
convention that completes the formalism of relativity.

What Is Measurable?
\begin{itemize}
\item The two-way (round-trip) speed of light: Measured directly by
  bouncing light off a mirror and recording round-trip time with a
  single clock.
\item Observables like time dilation, length contraction, Doppler
  shift, and light aberration -- all frame-independent or computable
  with Einstein synchronization.
\end{itemize}

Why All Experiments Agree

Every experiment that measures the speed of light (from
Michelson-Morley to modern cavity resonator setups) ultimately
measures the round-trip speed.  No known experiment can avoid the need
for clock synchronization to measure a one-way speed, and thus none
can determine it independently.

Conclusion

Yes, Einstein's synchronization convention defines the one-way speed
of light to be $c$.  Once you accept that postulate:

You've removed the one-way speed of light from the list of things you
can empirically measure.

It's not a weakness -- it's a recognition of how time and simultaneity
must be operationally defined in special relativity.

Let me know if you'd like examples or spacetime diagrams to visualize
this.


\subsection{If we reject Einstein's clock synchronizatino postulate, does it become possible to measure one way speed of light?}

\paragraph{I followed up a bit further about rejecting Einstein's postulate:}

Suppose one wished to reject Einstein's clock synchronization
postulate, and instead follow a theory such as Lorentz Ether Theory.
Does the one way speed of light then become a value that can be
discovered, rather than assumed?

\paragraph{ChaptGPT's reponse:}

Yes -- if you reject Einstein's synchronization convention and adopt a
framework like Lorentz Ether Theory (LET), then the one-way speed of
light becomes, in principle, a quantity that could be discovered, not
just assumed.

But this comes with important caveats -- both philosophical and
empirical.

In Lorentz Ether Theory (LET):
\begin{itemize}
\item There is a preferred ``ether'' frame in which light travels
  isotropically at speed $c$.
\item In all other frames, light travels anisotropically (i.e., with
  different speeds in different directions).
\item The asymmetry in the one-way speed of light between moving
  frames is considered physically real, not just a coordinate effect.
\item Clocks can, in principle, be synchronized to reveal this
  anisotropy -- if you could find the ether frame or distinguish it
  experimentally.
\end{itemize}
So under LET:

The one-way speed of light is not postulated to be $c$ in all
directions -- and could be discovered, if you can identify the ether
frame or detect the anisotropy.

But Here’s the Catch:

Despite this theoretical openness to a discoverable one-way speed of
light, LET makes all the same experimental predictions as Special
Relativity (SR) -- including:
\begin{itemize}
\item Michelson–Morley experiment: LET explains the null result by
  length contraction and time dilation due to motion through the
  ether.
\item Kennedy–Thorndike experiment
\item Ives–Stilwell experiment
\end{itemize}

Every experiment that tests light speed or time dilation gives results
indistinguishable from SR, because LET essentially hides the ether’s
effects perfectly.

This is what led Einstein and later physicists to regard LET as
superfluous -- philosophically heavier but empirically equivalent to
SR.

Why LET Still Can’t Measure the One-Way Speed of Light

Even in LET, if you're inside a lab moving with respect to the ether,
and all your measuring devices (clocks, rods) are affected by time
dilation and length contraction according to LET, you still can’t tell
that your one-way light speed is anisotropic -- because:
\begin{itemize}
\item Your rods have contracted exactly enough,
\item Your clocks have slowed down exactly enough,
\item So that all your measurements appear as if light speed is $c$ in
  all directions.
\end{itemize}
So unless you can somehow escape or detect the ether frame (which has
never been done), you still can’t operationally determine the one-way
speed of light.

Summary

\begin{tabularx}{1.0\textwidth}{|p{3.4cm}|p{3.7cm}|p{2.8cm}|p{3.6cm}|}
\cline{1-4} \\
Framework & One-way speed of light: & Can be measured? & Requires synchronization convention?
\\ \cline{1-4}
Special Relativity & Postulated $= c$ & No & Yes (Einstein convention)
\\ \cline{1-4}
Lorentz Ether Theory & Anisotropic except in ether frame & No, in practice & Yes, but defined by real ether time
\\ \cline{1-4}
If ether frame is found & Then: discoverable & Yes, in principle & No (objective simultaneity exists)
\\ \cline{1-4}
\end{tabularx}
So yes: in principle, LET treats the one-way speed of light as
discoverable -- but in practice, due to the empirical equivalence with
SR and the undetectability of the ether, it remains unmeasurable.

Would you like to explore experimental setups that attempt to measure
the one-way speed of light -- and why they fail -- even from a LET
perspective?


\subsection{Experimental setups that attempt to measure the one way speed of light, and why they fail, even from LET perspective}

\paragraph{My next prompt:}

Please explore experimental setups that attempt to measure the
one-way-speed of light, and why they fail, even from a LET
perspective.

\paragraph{ChaptGPT's reponse:}

Great -- let's dig into this.

You're asking about experimental attempts to measure the one-way speed
of light, and especially why they fail, even under the assumption of a
preferred frame like in Lorentz Ether Theory (LET).

This gets at a subtle point in both philosophy of physics and
measurement theory: why clock synchronization, or the absence of a
universal simultaneity, makes the one-way speed inaccessible, no
matter how clever the setup.

General Idea Behind One-Way Light Speed Experiments

To measure the one-way speed of light from point A to point B, you
need:
\begin{enumerate}
\item A clock at A to mark the light's departure time.
\item A clock at B to mark the arrival time.
\item The clocks must be synchronized in a physically meaningful way
  that does not assume the very light speed you're trying to measure.
\end{enumerate}
This third requirement is where every setup fails -- either
explicitly, by assuming Einstein synchronization, or implicitly, by
using signals whose behavior depends on light's speed.

Categories of Proposed Experiments (and Why They Fail)

1. Einstein Synchronization Itself

Method: Send a light signal from A to B and back to A.  Assume it
travels the same speed both ways, so the midpoint of round-trip time
is when it hit B.

Why it fails: It defines the one-way speed of light to be equal in
both directions.  So using it to measure the one-way speed is
circular.

2. Slow Clock Transport

Method: Synchronize clocks at the same location, then slowly move one
to the other location.

Why it fails:
\begin{itemize}
\item In LET: the transported clock experiences time dilation due to
  movement through the ether, even if slow.
\item This makes the result frame-dependent, and if the ether is
  undetectable, so is the effect.
\item In SR: same problem -- the result depends on assumptions about
  simultaneity and inertial frames.
\end{itemize}
So slow transport can't determine whether any observed discrepancy is
due to clock transport or actual light anisotropy.

3. Use of ``Clock-Free'' Methods (e.g., one-arm interferometers)

Example: A light signal is sent from A to B, reflected back, but with
a detector that moves to record the one-way leg specifically.

Why it fails: Even if you cleverly design a setup to record only the
forward travel, you still rely on:
\begin{itemize}
\item Assumptions about timing mechanisms,
\item Mechanical or electromagnetic signals traveling at predictable
  speeds,
\item Which ultimately trace back to synchronization based on light or
  some equivalent.
\end{itemize}
So you haven't escaped the need for synchronized clocks or signal
calibration that presumes isotropy.

4. Atomic or GPS-like Time Transfer

Method: Use portable atomic clocks, or time-transfer protocols (like
those in GPS) to synchronize clocks independently of light.

Why it fails:
\begin{itemize}
\item All practical time-transfer methods either rely on round-trip
  signals (so the same symmetry assumption sneaks in), or require
  extremely precise control over motion and knowledge of all inertial
  effects.
\item GPS works by assuming SR's validity -- not testing it.  In fact,
  GPS is designed assuming the one-way speed of light is isotropic in
  ECI (Earth-Centered Inertial) frame.
\end{itemize}
So if you try to use GPS to ``measure'' the one-way speed, you're
assuming what you want to test.

5. Two Clocks Launched in Opposite Directions

Method: Launch two synchronized clocks from a midpoint in opposite
directions (e.g., rockets or satellites).  They both receive the same
light pulse from the midpoint.

Why it fails:
\begin{itemize}
\item Time dilation affects both clocks differently depending on their
  motion.
\item Again, interpreting the results requires assuming a simultaneity
  convention.
\item LET would say the different readings are due to motion through
  the ether; SR would attribute it to coordinate-dependent
  simultaneity.  But the observations are the same.
\end{itemize}

What LET Says About All This

In Lorentz Ether Theory, there's a preferred absolute frame in which
the one-way speed of light is isotropic.  But:
\begin{itemize}
\item Clocks moving through the ether tick slower,
\item Rods contract in the direction of motion,
\item All measuring devices conspire to hide the anisotropy of light's
  one-way speed.
\end{itemize}
This ``conspiracy'' is not mystical -- it's a mathematical consequence
of the transformation rules that LET inherits from Lorentz
transformations (just like SR does).

So any experiment done with physical clocks and rods, no matter how
precise, will give the same result predicted by SR.  You can postulate
a preferred frame, but you can't detect it.

Bottom Line

Why all known attempts fail (even in LET):
\begin{itemize}
\item You always need synchronized clocks at two locations.
\item Synchronization requires assumptions -- either about signal
  speeds or time transport.
\item LET allows for a true one-way speed of light -- but makes it
  experimentally undetectable, due to the way time dilation and length
  contraction perfectly hide the ether.
\end{itemize}

What Would It Take to Measure the One-Way Speed?

You'd need:
\begin{enumerate}
\item A way to synchronize distant clocks without using light or slow
  transport or any system affected by ether motion.
\item Or, detect physical effects that vary with light directionality
  and cannot be explained by time dilation or length contraction --
  but no such effects have ever been observed.
\end{enumerate}

Closing Thought

Let's quote Einstein himself (paraphrased from his 1905 paper):

``We assume the speed of light to be the same in both directions.
That is a stipulation... not a hypothesis about the nature of light.''

This is as true in Special Relativity as it is as a criticism of any
experiment claiming to measure the one-way speed.

Let me know if you'd like a diagrammatic explanation, or an overview
of how this plays into the design of systems like GPS, or into the
philosophy of simultaneity.


\subsection{What would anisotropic speeds of light in Lorentz Ether Theory look like?}

\paragraph{The discussion above and my own investigations led me to ask this:}

Theoretically, what would anisotropic speeds of light in different
directions look like when moving relative to the ether in Lorentz
ether theory?

\paragraph{ChaptGPT's reponse:}

Excellent question.  You're now asking:

If Lorentz Ether Theory (LET) is correct, and there is a preferred
``ether frame,'' what would anisotropic one-way speeds of light look
like in a frame moving relative to the ether?

This dives into how LET conceptualizes the behavior of light and time
across different frames -- and how it agrees numerically with Special
Relativity while maintaining a different interpretive stance.


Basic Setup in Lorentz Ether Theory

\begin{itemize}
\item LET postulates a preferred frame -- the ``ether frame'' -- in
  which the laws of physics take their simplest form, and the speed of
  light is isotropic (same in all directions), equal to $c$.
\item All other inertial frames move relative to the ether.
\item Objects and clocks in moving frames undergo length contraction
  and time dilation when moving through the ether, as real physical
  effects.
\item The coordinate transformations between frames are still Lorentz
  transformations -- just with a hidden preferred frame underneath.
\end{itemize}

What Happens in a Frame Moving at Velocity $\vect{v}$ Through the Ether?

Suppose frame $S'$ moves at velocity $\vect{v}$ relative to the ether
frame $S$.  In $S$, light moves at speed $c$ in all directions.

In $S'$, due to LET:
\begin{itemize}
\item Light still takes the same total round-trip time over any path
  (because clocks and rods are distorted in just the right way).
\item BUT the one-way speed of light becomes anisotropic in $S'$.
\end{itemize}

Anisotropic Light Speed in LET

Let $\hatvec{n}$ be a unit vector representing direction, and let
$\vect{v}$ be the velocity of the moving frame through the ether (both
3D vectors).

Then, in LET, the one-way speed of light in direction $\hatvec{n}$ is
given by:
\begin{equation}
c_{one-way}(\hatvec{n}) = \frac{c^2}{c + \vect{v}\cdot\hatvec{n}}
\end{equation}
Similarly, in the opposite direction $-\hatvec{n}$, it becomes:
\begin{equation}
c_{one-way}(-\hatvec{n}) = \frac{c^2}{c - \vect{v}\cdot\hatvec{n}}
\end{equation}
This is assuming you're using ``true time'' and ``true rods'' (i.e.,
from the ether frame), and measuring what the moving observer would
perceive as the one-way speed.

Note:
\begin{itemize}
\item When $\vect{v} \cdot \hatvec{n} = 0$: the direction is
  perpendicular to motion through the ether → speed = $c$
\item When $\hatvec{n}$ is in the direction of motion, the speed is
  less than $c$.
\item When $\hatvec{n}$ is opposite to motion, the speed is greater
  than $c$.
\end{itemize}
But since LET includes length contraction, time dilation, and
asynchronous clocks, all these anisotropies are undetectable with
standard equipment.

Derivation Sketch (From Lorentz Transformations)

\begin{enumerate}

\item Light in the ether frame obeys:
\begin{equation}
x = ct
\end{equation}

\item Transform to the moving frame $S'$ (moving at velocity $v$ in
  the $x$-direction):
\begin{equation}
x' = \gamma(x-vt), t' = \gamma(t-vx/c^2)
\end{equation}

\item Solve for the apparent speed $dx'/dt'$, and you find:
\begin{equation}
\frac{dx'}{dt'} = \frac{c-v}{1-(vc)/c^2} = \text{anisotropic}
\end{equation}
\end{enumerate}
This calculation shows how even though light moves at speed $c$ in the
ether frame, the transformed coordinates show direction-dependent
speeds in the moving frame.

Visual/Physical Interpretation

Imagine an observer in a spaceship moving through the ether:
\begin{itemize}
\item If they try to measure the time it takes light to travel forward
  vs backward using clocks at both ends, they'll measure equal times,
  because their clocks are desynchronized in just the right way (due
  to Lorentz transformations).
\item But in ``real'' terms (from the ether perspective), the light
  really did go slower in the forward direction and faster in the
  backward one.
\end{itemize}
In this view:
\begin{itemize}
\item The one-way speed is anisotropic,
\item But the two-way speed is always $c$.
\item And the instruments conspire to hide the anisotropy.
\end{itemize}

Philosophical Takeaway

This is where LET and SR diverge philosophically but not
experimentally:

\begin{tabularx}{1.0\textwidth}{|p{4.5cm}|p{4.5cm}|p{4.5cm}|}
\cline{1-3} \\
Feature & Special Relativity & Lorentz Ether Theory
\\ \cline{1-3}
Speed of light (one-way) & Defined isotropic by convention & Isotropic {\em only} in ether frame
\\ \cline{1-3}
Simultaneity & Relative and conventional & Absolute but hidden
\\ \cline{1-3}
Time dilation & Symmetric between observers & Real physical effect due to motion through ether
\\ \cline{1-3}
Preferred frame & None & Exists but undetectable
\\ \cline{1-3}
Experimental predications & Identical & Identical
\\ \cline{1-3}
\end{tabularx}
LET says: There is a ``true'' frame, and light is only really
isotropic in that frame -- but our rulers and clocks deform in such a
way that we can never prove this.

Summary

\begin{itemize}
\item In Lorentz Ether Theory, the one-way speed of light in a frame
  moving at velocity $\vect{v}$ through the ether becomes anisotropic.
\item The apparent one-way speed in direction $\hatvec{n}$ is:
\begin{equation}
c_{one-way}(\hatvec{n}) = \frac{c^2}{c + \vect{v}\cdot\hatvec{n}}
\end{equation}
\item This anisotropy is perfectly compensated by physical effects
  (time dilation, length contraction, clock desynchronization), making
  it undetectable.
\item Thus, LET and SR yield identical experimental predictions, but
  differ in interpretation.
\end{itemize}

Would you like a diagram showing this anisotropy visually, or a code
snippet to simulate light paths with anisotropic speed?


\subsection{Correcting something in ChatGPT's previous answer}

\paragraph{I checked the calculation in the previous answer for $dx'/dt'$, and found it is always exactly $c$, which seemed not to be what it was intended to show, which was a speed other than $c$:}

The Lorentz transform sketch you gave showed that
$dx'/dt' = (c-v)/(1-(vc)/c^2)$, which is exactly $c$.
That does not seem to demonstrate what you wanted to show.

\paragraph{ChaptGPT's reponse:}

You're absolutely right -- excellent catch.

The expression
\begin{equation}
\frac{dx'}{dt'} = \frac{c-v}{1-(vc)/c^2} = c
\end{equation}
is indeed exactly equal to $c$, so it doesn't show anisotropy at all.
That was a misstep in the sketch -- it's just the standard Lorentz
transformation preserving the speed of light for light in any
direction (as expected in Special Relativity).  So you're right to
point out: it fails to demonstrate what Lorentz Ether Theory (LET)
uniquely claims -- anisotropy of one-way light speed in moving frames.

Let's now correctly approach the anisotropy of the one-way speed of
light in LET, and distinguish this from what Lorentz transformations
show in Einstein's interpretation (SR).

Correct Perspective: LET's View of One-Way Light Speed

In Lorentz Ether Theory, the key is to not assume Einstein
synchronization of clocks in the moving frame.  Instead, clocks are
synchronized according to ``true'' simultaneity defined in the ether
frame, and this is what reveals anisotropy.

Setup:
\begin{itemize}
\item Ether frame: $S$
\item Moving frame: $S'$, moving at velocity $\vect{v}$ relative to
  the ether
\item Light emitted from the origin at $t=0$ in all directions
\end{itemize}

From the ether frame's perspective, light travels outward spherically
at speed $c$.  But observers in $S'$ are moving, and their rulers and
clocks are affected by length contraction and time dilation in just
the right way to hide anisotropy if Einstein synchronization is used.

If instead they use absolute simultaneity from the ether frame to
synchronize clocks, then one-way light speed becomes anisotropic in
$S'$.

Actual Anisotropic Speed in LET

Let's derive the apparent one-way speed of light in the moving frame
$S'$, for light traveling in some direction $\hatvec{n}$, using
absolute simultaneity (i.e. clocks are synchronized by the ether
frame’s simultaneity).

Let:
\begin{itemize}
\item $\vect{v} =$ velocity of $S'$ through the ether
\item $\hatvec{n} =$ unit vector in direction light is traveling
\end{itemize}
Then the one-way speed of light in $S'$, using ether-defined
simultaneity, is:
\begin{equation}
c'(\hatvec{n}) = c - \vect{v} \cdot \hatvec{n}
\end{equation}
This is {\em not} relativistic velocity addition -- it is simply
expressing that if the ether frame sees light move at speed $c$ and
the moving observer travels at velocity $\vect{v}$, then the observer
in $S'$, with clocks synchronized according to the ether, will observe
light approaching faster or slower depending on the direction:
\begin{itemize}
\item In direction of motion ($\hatvec{n} = \hatvec{v}$): speed is
  $c-v$
\item Against direction of motion ($\hatvec{n} = -\hatvec{v}$): speed
  is $c+v$
\item Perpendicular ($\vect{v} \cdot \hatvec{n} = 0$): speed is $c$
\end{itemize}
This is the LET picture of anisotropy of one-way light speed.

Why Lorentz Transformations Still Show $c$

The earlier sketch used Lorentz transformations to transform
coordinates of light motion -- but those transformations assume
Einstein synchronization in every frame.

So when you apply the Lorentz transform to light paths from the ether
frame into $S'$, you always get light traveling at speed $c$ in all
directions, because Einstein's postulate is baked into the
transformation.

That's why:
\begin{itemize}
\item In Special Relativity: light speed is always $c$, even after
  transforming between frames.
\item In Lorentz Ether Theory: light speed is {\em really} only
  isotropic in the ether frame; the observed anisotropy in other
  frames is masked unless you use absolute simultaneity to reveal it.
\end{itemize}

Summary

\begin{itemize}

\item You're right: the earlier expression showing $dx'/dt' = c$ is
  correct only when using Einstein-synchronized clocks, and shows no
  anisotropy.

\item To see anisotropy, you must drop Einstein synchronization and
  use the ether-defined simultaneity, which is not built into the
  Lorentz transformation.

\item In Lorentz Ether Theory, the one-way speed of light in a frame
  moving at velocity $\vect{v}$ through the ether is:
\begin{equation}
c'(\hatvec{n}) = c - \vect{v} \cdot \hatvec{n}
\end{equation}

\item But this anisotropy is undetectable experimentally, because
  Einstein synchronization erases it -- hence LET and SR make the same
  predictions.
\end{itemize}

Would you like to explore how this anisotropy affects time
synchronization and simultaneity surfaces in LET?


\subsection{Is Einstein's clock synchronization postulate baked into the Lorentz transform?}

Above ChatGPT made this statement:

``So when you apply the Lorentz transform to light paths from the
ether frame into $S'$, you always get light traveling at speed $c$ in
all directions, because Einstein's postulate is baked into the
transformation.''

\paragraph{I decided to follow up on the statement from ChatGPT above with this prompt:}

How are is Einstein's postulate of clock synchronization baked into
the Lorentz tranformation?

\paragraph{ChaptGPT's reponse:}

This is a very deep and subtle question, and you're right to probe it.
The short answer is:

The Lorentz transformation is constructed in such a way that it
assumes the Einstein synchronization convention in every inertial
frame.

Let's unpack what this really means, step by step.

1. Einstein's Synchronization Convention

Einstein's 1905 method for synchronizing distant clocks is:
\begin{enumerate}
\item Send a light signal from clock A to clock B.
\item Assume the light travels at the same speed in both directions
  (one-way speed = $c$).
\item Set the time on B’s clock such that:
  \begin{equation}
  t_B = t_A + \frac{1}{2} (t_{A'} - t_A)
  \end{equation}
  where $t_A$ is the send time, $t_{A'}$ is the receive time of the
round-trip signal at A.
\end{enumerate}
This convention defines simultaneity in each frame, by assuming
isotropic light speed.

You can't measure one-way light speed without synchronized clocks --
and you can't synchronize clocks without assuming something about
one-way light speed.  Einstein's method breaks this circularity by
defining both together.

2. Lorentz Transform: What It Encodes

The Lorentz transformation between two frames $S$ and $S'$, where $S'$
moves at velocity $\vect{v}$ along the $x$-axis relative to $S$, is:
\begin{align*}
x' & = \gamma (x - vt) \\
t' & = \gamma (t - \frac{vx}{c^2})
\end{align*}
where $\gamma = \frac{1}{\sqrt{1 - v^2/c^2}}$.

The Key Thing:

The Lorentz transformation preserves the form of the spacetime
interval:
\begin{equation}
ds^2 = c^2dt^2 - dx^2 - dy^2 - dz^2
\end{equation}
and guarantees that light always moves at speed $c$ in all directions,
in all inertial frames, when you use $(x', t')$ as your new
coordinates.

This is only true if:
\begin{itemize}
\item You define time $t'$ using the same synchronization procedure
  (Einstein's) in each frame.
\end{itemize}

3. Where Synchronization Is Hidden in the Lorentz Transform

Look at the transformed time coordinate:
\begin{equation}
t' = \gamma(t - \frac{vx}{c^2})
\end{equation}
This term $\frac{vx}{c^2}$ introduces a position-dependent offset in
time -- it shifts simultaneity in a way that makes light travel at
speed $c$ in $S'$.

This shift encodes Einstein synchronization.

Let's explain how.

4. What If You Used a Different Synchronization?

Suppose instead of Einstein's convention, you used a more general
clock synchronization in $S'$.  For example, the Reichenbach epsilon
convention:
\begin{equation}
t_B = t_A + \epsilon (t_{A'} - t_A)
\end{equation}
with $0 < \epsilon < 1$, where Einstein's method assumes
$\epsilon = 1/2$.

Then your coordinate transformation from $S$ to $S'$ would look
different -- the time coordinate in $S'$ would no longer be:
\begin{equation}
t' = \gamma (t - \frac{vx}{c^2})
\end{equation}
but something more general, perhaps:
\begin{equation}
t' = \gamma (t - \frac{v (1 - 2\epsilon) x}{c^2})
\end{equation}
This breaks the symmetry that makes the one-way speed of light the
same in all directions.

So:

The form of the Lorentz transformation assumes clocks in every frame
are synchronized using Einstein’s method, i.e., so that light’s
one-way speed = $c$.

5. In Contrast: LET's Viewpoint

In Lorentz Ether Theory, the ``true'' frame (the ether frame) has real
simultaneity, and any other frame is moving relative to that.

In that view, clocks in moving frames are not truly synchronized when
using Einstein's convention -- they're conventionally synchronized in
a way that masks the anisotropy of light's one-way speed.

So from the LET perspective:

The Lorentz transformation doesn't describe ``reality'' -- it
describes what you would measure if you synchronized clocks as
Einstein prescribes, which conceals the anisotropy due to motion
through the ether.

Conclusion: How Is Einstein's Synchronization Built In?

\begin{tabularx}{1.0\textwidth}{|p{3.4cm}|p{10cm}|}
\cline{1-2} \\
Element & How it shows Einstein synchronization
\\ \cline{1-2}
Lorentz time transformation & $t' = \gamma(t - \frac{vx}{c^2})$ contains position-dependent shift that enforces Einstein simultaneity
\\ \cline{1-2}
Isotropy of light speed in all frames & Guaranteed only if Einstein clock sync is used
\\ \cline{1-2}
Form invariance of wave equation & Requires Einstein synchronization
\\ \cline{1-2}
Failure to detect anisotropy & Only holds under Einstein synchronization
\\ \cline{1-2}
\end{tabularx}

In short: the Lorentz transformation is constructed such that every
inertial observer sees light travel at speed $c$ in all directions,
and this only works if clocks are synchronized according to Einstein's
convention.

That's what ``baking in'' the synchronization means.

Would you like to see what the Lorentz transformation would look like
if you used a non-Einstein synchronization convention, like
Reichenbach's $\epsilon$-scheme?

\paragraph{Take advantage of the offer made by ChatGPT:}

Yes.

\paragraph{ChaptGPT's reponse:}

Great!  Let's construct the generalized Lorentz transformation using a
non-Einstein synchronization convention, often called the Reichenbach
$\epsilon$-convention.

1. What Is the $\epsilon$-Synchronization Convention?

In Einstein synchronization, we assume the one-way speed of light is
equal in both directions, so:
\begin{itemize}
\item Light travel time from A to B = travel time from B to A.
\item This corresponds to setting:
  \begin{equation}
    \epsilon = 1/2
  \end{equation}
\end{itemize}
The Reichenbach $\epsilon$-convention generalizes this by defining
simultaneity asymmetrically:
\begin{itemize}
\item When clock A sends a light signal at time $t_1$, and receives
  the return at time $t_3$, we say that the event of reception at B
  happened at:
  \begin{equation}
    t_2 = t_1 + \epsilon (t_3 - t_1)
  \end{equation}
  with $0 < \epsilon < 1$.
\end{itemize}
Thus:
\begin{itemize}
\item If $\epsilon < 1/2$: Light is faster from A to B than B to A.
\item If $\epsilon > 1/2$: Light is slower from A to B than B to A.
\end{itemize}
This is a coordinate choice, not a physical difference -- all values
of $\epsilon$ produce the same measurable outcomes.

2. Generalized Lorentz Transformation with $\epsilon$-Synchronization

We now modify the Lorentz time transformation to reflect the fact that
clocks are not synchronized using Einstein's method.

The modified transformation becomes:
\begin{align*}
  x' & = \gamma (x - vt) \\
  t' & = \gamma (t - \frac{vx}{c^2} (1 - 2 \epsilon))
\end{align*}
Compare this with the standard Lorentz transform (Einstein
synchronization, $\epsilon = 1/2$):
\begin{align*}
  t' & = \gamma (t - \frac{vx}{c^2})
\end{align*}
So:

The only change is that $\frac{vx}{c^2}$ becomes
$\frac{vx}{c^2}(1-2\epsilon)$.

This means:
\begin{itemize}
\item When $\epsilon = 1/2$, you recover Einstein synchronization.
\item When $\epsilon \neq 1/2$, the time coordinate is shifted to
  reflect the asymmetric synchronization choice.
\end{itemize}

3. What Does This Do Physically?

This shift changes what events are considered simultaneous in the
primed frame:
\begin{itemize}
\item Different values of $\epsilon$ define different simultaneity
  slices.
\item It does not change what anyone measures -- round-trip speeds of
  light are still $c$, time dilation and length contraction remain the
  same.
\end{itemize}
But it does alter:
\begin{itemize}
\item The coordinate velocity of light in one direction vs the other.
\item The coordinate time assigned to distant events.
\end{itemize}

4. Coordinate Speed of Light Becomes Direction-Dependent

If a light signal travels in the $x$-direction in $S'$, the coordinate
speed becomes:

\begin{equation}
c' = \frac{dx'}{dt'} = \frac{c-v}{1 - v(1-2\epsilon)/c}
\end{equation}
This depends on $\epsilon$.

For example:
\begin{itemize}
\item If $\epsilon = 0$, then $t' = \gamma(t - \frac{vx}{c^2})$ shifts
  more than Einstein's.
\item If $\epsilon = 1$, the shift goes the other way.
\end{itemize}

Only for $\epsilon = 1/2$ does $c' = c$ in both directions.

Thus, the one-way speed of light is only isotropic if $\epsilon =
1/2$.

5. Summary

\begin{tabularx}{1.0\textwidth}{|p{3.4cm}|p{5cm}|p{5cm}|}
\cline{1-3} \\
Synchronization Convention & Time Transformation & One-way speed of light
\\ \cline{1-3}
Einstein ($\epsilon = 1/2$) & $t' = \gamma (t - \frac{vx}{c^2})$ & Isotropic (same both ways)
\\ \cline{1-3}
Reichenbach $\epsilon$ & $t' = \gamma (t - \frac{vx}{c^2} (1 - 2\epsilon)$ & Anisotropic (depends on direction)
\\ \cline{1-3}
\end{tabularx}

So:

Einstein's synchronization is just one choice among many valid
conventions -- but it's the one that makes the Lorentz transformation
clean and symmetric, and that makes the one-way speed of light come
out the same in every direction.

Optional: Minkowski Geometry View

The change in synchronization affects the tilt of simultaneity planes
(slices of constant time) in spacetime diagrams.

Different values of $\epsilon$ result in different choices of how to
foliate spacetime.

Would you like a spacetime diagram showing how simultaneity planes
tilt with different $\epsilon$ values?


\end{document}
